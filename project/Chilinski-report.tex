% 
\RequirePackage{amsmath}
\documentclass[a4paper]{article}
\usepackage{Sweave}
\usepackage[margin=0.3in]{geometry}
\usepackage{enumitem}
\usepackage{float}
\usepackage[usenames,dvipsnames]{color}
\usepackage{rotating}
\usepackage{longtable}
\usepackage{hyperref}
\hypersetup{%
    pdfborder = {0 0 0}
}
\usepackage[all]{hypcap}
  
\DefineVerbatimEnvironment{RoutputTight}{Verbatim}{fontsize=\tiny,
xleftmargin=-0.5cm}

\title{Advanced Statistical Methods - Project.}
\author{Pawel Chilinski}

\begin{document}
\input{Chilinski-report-concordance}

\maketitle

\tableofcontents

\listoffigures

\listoftables

\section{Introduction}
This project performs exercise of correlational(observational) studies. We have
data (predictors) which was gathered by the The World Bank \cite{worldbankData} and the predicted
variable (Corruption.Index) Failed States Index computed by the United
States think-tank Fund for Peace \cite{WikiFailedStatesIndex}. Because of the
nature of the data we cannot make strong causal conclusions on how predictors
influence the predicted variable (because of possible lurking variables). I
assume that we deal with the simple random sample.
The data consists of 31 predictor variables which gives
$2^{31}=$2 147 483 648 different models that can be fitted.

\section{Data}


Description of columns and data types:
\begingroup
\begin{table}[H]
\fontsize{7pt}{8pt}\selectfont
\begin{center}
  \advance\leftskip-0.5cm
  \begin{tabular}{ l | p{1.5cm} p{3cm} p{10cm} }
    \textbf{Variable} & \textbf{Type} & \textbf{Name} & \textbf{Description} \\
    \hline country & nominal & country & Country for which row contains various
    variables\\
    AG.LND.AGRI.K2 & ratio & Agricultural land (sq. km) & Agricultural land
    refers to the share of land area that is arable, under permanent crops, and under permanent pastures.\\
    AG.LND.ARBL.HA.PC & ratio & Arable land (hectares per person) & Arable
land (hectares per person) includes land defined by the FAO as land under temporary crops ,temporary meadows for mowing or for pasture,
land under market or kitchen gardens, and land temporarily fallow.  Land
abandoned as a result of shifting cultivation is excluded. \\
    AG.LND.ARBL.ZS & counted fraction & Arable land (\% of land area) & \% of
    land area which is Arable land \\ 
    AG.LND.CROP.ZS & counted fraction & Permanent cropland (\% of land area) & A permanent crop is one produced from plants which last for many seasons, rather
than being replanted after each harvest.\\
    AG.LND.TOTL.K2& ratio & Land area (sq. km) & Land area is a country's
total area.\\
    AG.PRD.CROP.XD& ratio & Crop production index (2004-2006 = 100) & Crop
production index shows agricultural production for each year relative to the base period 2004-2006. 
It includes all crops except fodder crops.\\
    AG.PRD.FOOD.XD& ratio & Food production index (2004-2006 = 100) & Food
production index covers food crops that are considered edible and that contain nutrients. 
Coffee and tea are excluded because, although edible, they have no nutritive value.\\
    AG.PRD.LVSK.XD& ratio & Livestock production index (2004-2006 = 100) & Livestock production index includes meat and milk from all sources, dairy products such as cheese, and eggs, honey, raw silk, wool, 
and hides and skins.\\
    AG.SRF.TOTL.K2& ratio & Surface area (sq. km) & Surface area is a
country's total area, including areas under inland bodies of water and some coastal waterways.\\
    AG.YLD.CREL.KG& ratio & Cereal yield (kg per hectare) & Cereal yield
    measured as kilograms per hectare of harvested land, includes wheat, rice, maize, barley, oats, rye, millet, sorghum, buckwheat, and mixed grains.\\
    BM.GSR.INSF.ZS& counted fraction & Insurance and financial services (\% of
service imports, \% BoP) & Insurance and financial services cover various types of insurance
provided to nonresidents by resident insurance enterprises and vice versa, and financial intermediary and auxiliary services 
(except those of insurance enterprises and pension funds) exchanged between residents and nonresidents.\\
    BM.GSR.TRVL.ZS& counted fraction & Travel services (\% of service imports,
BoP) & Travel covers goods and services acquired from an economy by travelers for their own use during visits of less than one year in that economy for either 
business or personal purposes.\\
    BX.GSR.CMCP.ZS& counted fraction& Communications, computer, etc. (\% of
service exports, \% BoP) & Communications, computer, information, and other services cover
international telecommunications; computer data; news-related service transactions between residents and nonresidents; 
construction services; royalties and license fees; miscellaneous business, professional, and technical services; personal, 
cultural, and recreational services; manufacturing services on physical inputs owned by others; and maintenance and repair services and 
government services not included elsewhere.\\
    BX.KLT.DINV.WD.GD.ZS& counted fraction & Foreign direct investment, net
inflows (\% of GDP) & Foreign direct investment are the net inflows of
investment to acquire a lasting management interest (10 percent or more of voting stock) in an enterprise operating in an economy other than that of the investor.\\
    EG.GDP.PUSE.KO.PP& ratio & GDP per unit of energy use (PPP \$ per kg of
oil equivalent) & GDP per unit of energy use is the PPP GDP per kilogram of oil
equivalent of energy use.\\
    EG.GDP.PUSE.KO.PP.KD& ratio & GDP per unit of energy use (constant 2005
PPP \$ per kg of oil equivalent) & \\
    EG.USE.COMM.KT.OE& ratio & Energy use (kt of oil equivalent) & Energy
use refers to use of primary energy before transformation to other end-use fuels, which is equal to indigenous production plus imports and stock changes, 
minus exports and fuels supplied to ships and aircraft engaged in international transport.\\
    EG.USE.COMM.GD.PP.KD& ratio & Energy use (kg of oil equivalent) per
\$1,000 GDP (constant 2005 PPP) & Energy use per PPP GDP is the kilogram of oil equivalent
of energy use per constant PPP GDP.\\
    EG.USE.ELEC.KH.PC& ratio & Electric power consumption (kWh per capita) & Electric power consumption measures the production of power plants and combined heat and power plants less transmission, distribution, 
and transformation losses and own use by heat and power plants.\\
    EN.ATM.CO2E.KD.GD& ratio & CO2 emissions (kg per 2005 US\$ of GDP) & \\
    EN.ATM.CO2E.PC& ratio & CO2 emissions (metric tons per capita)  &  Carbon
dioxide emissions are those stemming from the burning of fossil fuels and the manufacture of cement. They include carbon dioxide produced during consumption of solid, 
liquid, and gas fuels and gas flaring.\\
    EN.ATM.PM10.MC.M3& ratio & PM10, country level (micrograms per cubic
meter) & Particulate matter concentrations refer to fine suspended particulates less than 10 microns in diameter (PM10) that are capable of penetrating 
deep into the respiratory tract and causing significant health damage.\\
    ER.H2O.INTR.K3& ratio & Renewable internal freshwater resources, total
(billion cubic meters) & Renewable internal freshwater resources flows refer to internal
renewable resources (internal river flows and groundwater from rainfall) in the country.\\
    ER.H2O.INTR.PC& ratio & Renewable internal freshwater resources per capita
(cubic meters) & Renewable internal freshwater resources flows refer to internal renewable resources (internal river flows and groundwater from rainfall) in the country. 
Renewable internal freshwater resources per capita are calculated using the World Bank's population estimates.\\
    FM.LBL.MQMY.GD.ZS& counted fraction & Money and quasi money (M2) as \% of
GDP & Money and quasi money comprise the sum of currency outside banks, demand deposits other than those of the central government, and the time, savings, 
and foreign currency deposits of resident sectors other than the central government. This definition of money supply is frequently called M2; 
it corresponds to lines 34 and 35 in the International Monetary Fund's (IMF) International Financial Statistics (IFS).\\
    FS.AST.PRVT.GD.ZS& counted fraction & Domestic credit to private sector
(\% of GDP) & Domestic credit to private sector refers to financial resources provided to the private sector, such as through loans, 
purchases of nonequity securities, and trade credits and other accounts receivable, that establish a claim for repayment. 
For some countries these claims include credit to public enterprises.\\
    IC.CRD.PRVT.ZS& counted fraction & Private credit bureau coverage (\% of
adults) & Private credit bureau coverage reports the number of individuals or firms listed by a private credit bureau with current information on repayment history, 
unpaid debts, or credit outstanding. The number is expressed as a percentage of the adult population.\\
    IC.EXP.DURS& ratio & Time to export (days) & Time is recorded in calendar
days. The time calculation for a procedure starts from the moment it is
initiated and runs until it is completed.\\
    IC.LGL.CRED.XQ& ordinal & Strength of legal rights index (0=weak to
10=strong) & Strength of legal rights index measures the degree to which collateral and bankruptcy laws protect the rights of borrowers and lenders and thus facilitate lending. 
The index ranges from 0 to 10, with higher scores indicating that these laws are better designed to expand access to credit.\\
    NE.RSB.GNFS.ZS& counted fraction & External balance on goods and services
(\% of GDP) & External balance on goods and services (formerly resource balance) equals exports of goods and services minus imports of goods and services 
(previously nonfactor services).\\
    NE.TRD.GNFS.ZS& counted fraction & Trade (\% of GDP) & Trade is the sum
of exports and imports of goods and services measured as a share of gross domestic product.\\
    Corruption.Index& ratio & Failed States Index (World Corruption Index) &
    \href{http://en.wikipedia.org/wiki/List_of_countries_by_Failed_States_Index}{Definition}\\
  \end{tabular}
\end{center}
\end{table}
\endgroup 

\subsection{Validating the data.} 

The total land should be bigger the agricultural land (checking if some county
has agricultural bigger that total area):
\begin{Schunk}
\begin{Sinput}
> as.character(kaggle.data$country[kaggle.data$AG.LND.AGRI.K2 > kaggle.data$AG.LND.TOTL.K2])
\end{Sinput}
\begin{Soutput}
[1] "Macedonia"
\end{Soutput}
\end{Schunk}
Total area of country (including water area) should be bigger than its land
area:
\begin{Schunk}
\begin{Sinput}
> all(kaggle.data$AG.SRF.TOTL.K2 >= kaggle.data$AG.LND.TOTL.K2)
\end{Sinput}
\begin{Soutput}
[1] TRUE
\end{Soutput}
\end{Schunk}
Percent of financial services should be $\in (0,100)$, show countries and values
not fulfilling this:
\begin{Schunk}
\begin{Sinput}
> kaggle.data[kaggle.data$BM.GSR.INSF.ZS<0 | kaggle.data$BM.GSR.INSF.ZS>100,c("country","BM.GSR.INSF.ZS")]
\end{Sinput}
\begin{Soutput}
       country BM.GSR.INSF.ZS
1  Afghanistan      -2.552965
30     Eritrea      -2.552965
52        Laos      -2.552965
68       Qatar      -2.552965
\end{Soutput}
\end{Schunk}
The rest of the data seems semantically correct.
\\ 
Fixing the data:
\begin{Schunk}
\begin{Sinput}
> #so we can remove data for Macedonia
> kaggle.data<-kaggle.data[kaggle.data$AG.LND.AGRI.K2 <= kaggle.data$AG.LND.TOTL.K2,]
> #for Afganistan we can check on data.worldbank.org that in 2008 it had BM.GSR.INSF.ZS = 5.8457031725
> kaggle.data[kaggle.data$country=="Afghanistan","BM.GSR.INSF.ZS"]<-5.8457031725
> #because Eritrea and Laos has incorrect value for BM.GSR.INSF.ZS and 
> #we cannot fill it from the website we remove observation for this country
> kaggle.data<-kaggle.data[kaggle.data$country!="Eritrea" & kaggle.data$country!="Laos",]
> #It looks that data for Qatar contains incorrect values for few attributes, also
> #without cleaning the data for this country becomes outlier and influential
> #observation so removing it
> kaggle.data<-kaggle.data[kaggle.data$country!="Qatar",]
\end{Sinput}
\end{Schunk}

Describing the data, we can see that all values are reasonable and output
variable can be considered normally distributed for the requirements of the
linear regression model (the kurtosis and skewness are within limits):
\begingroup
    \fontsize{8pt}{9pt}\selectfont
\begin{Schunk}
\begin{Soutput}
                     var  n       mean         sd    median   trimmed       mad    min         max       range  skew kurtosis        se
country*               1 83      43.73      25.67     43.00     43.67     32.62   1.00       87.00       86.00  0.02    -1.28      2.82
AG.LND.AGRI.K2         2 83  397402.52  970121.41  47930.00 137680.34  70779.32  50.00  5344172.00  5344122.00  3.52    12.58 106484.66
AG.LND.ARBL.HA.PC      3 83       0.26       0.34      0.17      0.19      0.16   0.00        2.57        2.56  4.28    23.61      0.04
AG.LND.ARBL.ZS         4 83      15.39      13.82     11.98     13.38     13.22   0.06       56.17       56.12  1.18     0.81      1.52
AG.LND.CROP.ZS         5 83       5.03       8.25      1.53      3.18      1.86   0.02       48.96       48.94  2.89     9.82      0.91
AG.LND.TOTL.K2         6 83 1084101.95 2692013.81 183780.00 360613.61 257631.40 200.00 16381390.00 16381190.00  3.60    13.67 295486.90
AG.PRD.CROP.XD         7 83     123.82      19.71    120.00    121.55     14.83  90.00      195.00      105.00  1.13     1.37      2.16
AG.PRD.FOOD.XD         8 83     123.84      18.61    122.00    122.07     17.79  98.00      191.00       93.00  1.03     1.43      2.04
AG.PRD.LVSK.XD         9 83     125.90      19.91    127.00    125.13     25.20  95.00      177.00       82.00  0.27    -0.98      2.19
AG.SRF.TOTL.K2        10 83 1128515.06 2815532.57 185180.00 370524.18 258254.09 200.00 17098240.00 17098040.00  3.61    13.69 309044.85
AG.YLD.CREL.KG        11 83    3244.41    1781.31   2807.70   3023.22   1675.63 812.60     9631.90     8819.30  1.10     1.08    195.52
BM.GSR.INSF.ZS        12 83      11.68       9.54      9.30     10.12      5.52   0.75       61.87       61.11  2.75    10.14      1.05
BM.GSR.TRVL.ZS        13 83      27.43      12.77     25.54     26.68     11.83   2.51       71.02       68.51  0.67     0.84      1.40
BX.GSR.CMCP.ZS        14 83      39.49      21.58     35.46     37.90     21.25   4.23      100.00       95.77  0.66    -0.28      2.37
BX.KLT.DINV.WD.GD.ZS  15 83      17.28      58.08      6.13      8.50      4.69   0.11      524.88      524.77  8.07    67.34      6.37
EG.GDP.PUSE.KO.PP     16 83       7.83       4.07      7.02      7.49      4.20   1.44       18.48       17.04  0.70    -0.25      0.45
EG.GDP.PUSE.KO.PP.KD  17 83       7.13       3.65      6.24      6.78      3.27   1.33       19.10       17.77  0.91     0.42      0.40
EG.USE.COMM.KT.OE     18 83  115820.23  356607.03  13578.00  38497.00  19866.84  42.00  2336546.00  2336504.00  5.12    27.21  39142.71
EG.USE.COMM.GD.PP.KD  19 83     233.44     197.17    185.99    194.42     98.74  61.29     1219.64     1158.35  3.18    12.23     21.64
EG.USE.ELEC.KH.PC     20 83    4105.63    6815.31   2017.49   2675.50   1786.29  49.15    50067.10    50017.95  4.33    23.67    748.08
EN.ATM.CO2E.KD.GD     21 83       1.61       1.99      0.83      1.14      0.56   0.20       11.33       11.13  2.84     8.62      0.22
EN.ATM.CO2E.PC        22 83       5.19       5.34      3.64      4.27      4.28   0.02       24.33       24.31  1.46     1.81      0.59
EN.ATM.PM10.MC.M3     23 83      52.87      38.71     42.56     46.52     24.94   7.44      212.39      204.95  2.02     4.76      4.25
ER.H2O.INTR.K3        24 83     377.48     938.57     37.20    121.09     53.67   0.02     5418.00     5417.98  3.48    12.63    103.02
ER.H2O.INTR.PC        25 83   28200.77   78417.45   2574.03  10885.75   3274.11  25.65   590277.78   590252.13  5.24    31.61   8607.43
FM.LBL.MQMY.GD.ZS     26 83      76.90      77.86     57.03     63.97     38.51  14.52      636.51      621.99  4.77    29.98      8.55
FS.AST.PRVT.GD.ZS     27 83      66.33      58.91     44.78     56.47     37.29   1.90      319.47      317.57  1.88     4.26      6.47
IC.CRD.PRVT.ZS        28 83      26.18      36.07      3.30     20.49      4.89   0.00      100.00      100.00  1.06    -0.47      3.96
IC.EXP.DURS           29 83      26.14      17.48     21.00     23.30      7.41   5.00      102.00       97.00  2.02     4.80      1.92
IC.LGL.CRED.XQ        30 83       5.31       2.38      5.00      5.30      2.97   1.00       10.00        9.00  0.03    -1.19      0.26
NE.RSB.GNFS.ZS        31 83       2.76      14.71      1.37      1.60     11.59 -31.04       46.61       77.65  0.74     0.96      1.62
NE.TRD.GNFS.ZS        32 83     101.23      54.54     86.49     94.50     42.13  28.97      324.33      295.35  1.37     2.48      5.99
Corruption.Index      33 83      68.91      22.20     74.50     70.00     14.38  19.70      113.40       93.70 -0.50    -0.46      2.44
\end{Soutput}
\end{Schunk}
\endgroup


\begin{figure}[H]
\begin{center}
\advance\rightskip-0.5cm
\advance\leftskip-1cm
\includegraphics[width=8.5in,height=11in]{figures/fig-HIST.pdf}
\caption{Histograms of variables}
\end{center}
\end{figure}

\begin{figure}[H]
\advance\rightskip-0.5cm
\advance\leftskip-1cm
\includegraphics[width=8.5in,height=11in]{figures/fig-PAIRS.pdf}
\caption{Scatterplots between variables}
\label{PAIRS_PLOT}
\end{figure}
\begingroup
\renewcommand{\arraystretch}{0.5} % space between rows (1 standard)
\begin{sidewaystable}
\advance\leftskip-0.5cm
\fontsize{0.05cm}{1em}\selectfont{
\centering
\caption{Pearson correlations between variables (describing linear relationship) as first
number in a cell, Spearman's rank correlations between variables as second
number in a cell (used to find monotonic relationships) and permutation test as
third number in a cell (percent of permutations the has bigger/smaller r value),
red font is used for significant values.
Because predictors and explained variable distributions depart from the normal distributions (|skew|>3, |kurtosis|>10, examining histograms) I am using
Spearman's correlation test and coefficient additionally to the Pearson's one.}

\begin{tabular}{p{0.5cm}p{0.5cm}p{0.5cm}p{0.5cm}p{0.5cm}p{0.5cm}p{0.5cm}p{0.5cm}p{0.5cm}p{0.5cm}p{0.5cm}p{0.5cm}p{0.5cm}p{0.5cm}p{0.5cm}p{0.5cm}p{0.5cm}p{0.5cm}p{0.5cm}p{0.5cm}p{0.5cm}p{0.5cm}p{0.5cm}p{0.5cm}p{0.5cm}p{0.5cm}p{0.5cm}p{0.5cm}p{0.5cm}p{0.5cm}p{0.5cm}p{0.5cm}} & AG.LND.AGRI.K2& AG.LND.ARBL.HA.PC& AG.LND.ARBL.ZS& AG.LND.CROP.ZS& AG.LND.TOTL.K2& AG.PRD.CROP.XD& AG.PRD.FOOD.XD& AG.PRD.LVSK.XD& AG.SRF.TOTL.K2& AG.YLD.CREL.KG& BM.GSR.INSF.ZS& BM.GSR.TRVL.ZS& BX.GSR.CMCP.ZS& BX.KLT.DINV.WD.GD.ZS& EG.GDP.PUSE.KO.PP& EG.GDP.PUSE.KO.PP.KD& EG.USE.COMM.KT.OE& EG.USE.COMM.GD.PP.KD& EG.USE.ELEC.KH.PC& EN.ATM.CO2E.KD.GD& EN.ATM.CO2E.PC& EN.ATM.PM10.MC.M3& ER.H2O.INTR.K3& ER.H2O.INTR.PC& FM.LBL.MQMY.GD.ZS& FS.AST.PRVT.GD.ZS& IC.CRD.PRVT.ZS& IC.EXP.DURS& IC.LGL.CRED.XQ& NE.RSB.GNFS.ZS& NE.TRD.GNFS.ZS\\ AG.LND.AGRI.K2& & & & & & & & & & & & & & & & & & & & & & & & & & & & & & & &  \\ 
AG.LND.ARBL.HA.PC&\textcolor{Red}{+0.51}\textcolor{Red}{+0.58}\textcolor{Red}{+0.00}& & & & & & & & & & & & & & & & & & & & & & & & & & & & & & &  \\ 
AG.LND.ARBL.ZS&-0.04+0.04+0.41&+0.02\textcolor{Red}{+0.22}+0.37& & & & & & & & & & & & & & & & & & & & & & & & & & & & & &  \\ 
AG.LND.CROP.ZS&-0.19\textcolor{Red}{-0.38}\textcolor{Red}{+0.00}&\textcolor{Red}{-0.24}\textcolor{Red}{-0.42}\textcolor{Red}{+0.00}&+0.10\textcolor{Red}{+0.39}+0.16& & & & & & & & & & & & & & & & & & & & & & & & & & & & &  \\ 
AG.LND.TOTL.K2&\textcolor{Red}{+0.80}\textcolor{Red}{+0.92}\textcolor{Red}{+0.00}&\textcolor{Red}{+0.55}\textcolor{Red}{+0.51}\textcolor{Red}{+0.00}&-0.11-0.17+0.14&-0.20\textcolor{Red}{-0.56}\textcolor{Red}{+0.00}& & & & & & & & & & & & & & & & & & & & & & & & & & & &  \\ 
AG.PRD.CROP.XD&+0.06\textcolor{Red}{+0.32}+0.27&+0.09\textcolor{Red}{+0.28}+0.19&+0.01-0.04+0.43&-0.17\textcolor{Red}{-0.25}\textcolor{Red}{+0.04}&+0.13\textcolor{Red}{+0.29}+0.14& & & & & & & & & & & & & & & & & & & & & & & & & & &  \\ 
AG.PRD.FOOD.XD&+0.04\textcolor{Red}{+0.34}+0.34&-0.02+0.16+0.46&-0.12-0.12+0.13&-0.19-0.15\textcolor{Red}{+0.03}&+0.07\textcolor{Red}{+0.29}+0.24&\textcolor{Red}{+0.79}\textcolor{Red}{+0.78}\textcolor{Red}{+0.00}& & & & & & & & & & & & & & & & & & & & & & & & & &  \\ 
AG.PRD.LVSK.XD&-0.06+0.17+0.28&-0.15+0.02+0.08&-0.14-0.14+0.10&-0.14-0.04+0.09&-0.06+0.15+0.29&\textcolor{Red}{+0.37}\textcolor{Red}{+0.36}\textcolor{Red}{+0.00}&\textcolor{Red}{+0.75}\textcolor{Red}{+0.75}\textcolor{Red}{+0.00}& & & & & & & & & & & & & & & & & & & & & & & & &  \\ 
AG.SRF.TOTL.K2&\textcolor{Red}{+0.79}\textcolor{Red}{+0.92}\textcolor{Red}{+0.00}&\textcolor{Red}{+0.55}\textcolor{Red}{+0.51}\textcolor{Red}{+0.00}&-0.11-0.18+0.17&-0.20\textcolor{Red}{-0.56}\textcolor{Red}{+0.00}&\textcolor{Red}{+1.00}\textcolor{Red}{+1.00}\textcolor{Red}{+0.00}&+0.13\textcolor{Red}{+0.29}+0.16&+0.07\textcolor{Red}{+0.29}+0.26&-0.07+0.15+0.28& & & & & & & & & & & & & & & & & & & & & & & &  \\ 
AG.YLD.CREL.KG&+0.18+0.16+0.06&-0.05-0.01+0.37&+0.04+0.05+0.34&-0.15-0.19+0.07&+0.13\textcolor{Red}{+0.27}+0.12&-0.07-0.01+0.26&\textcolor{Red}{-0.22}-0.18\textcolor{Red}{+0.03}&\textcolor{Red}{-0.23}-0.17\textcolor{Red}{+0.01}&+0.13\textcolor{Red}{+0.27}+0.10& & & & & & & & & & & & & & & & & & & & & & &  \\ 
BM.GSR.INSF.ZS&+0.00+0.05+0.39&-0.07+0.00+0.23&+0.06+0.05+0.27&-0.17-0.04\textcolor{Red}{+0.03}&-0.03+0.03+0.46&-0.17-0.17\textcolor{Red}{+0.04}&-0.17-0.08+0.06&-0.01+0.12+0.45&-0.03+0.03+0.48&+0.06+0.01+0.28& & & & & & & & & & & & & & & & & & & & & &  \\ 
BM.GSR.TRVL.ZS&+0.16+0.18+0.08&\textcolor{Red}{+0.27}\textcolor{Red}{+0.29}\textcolor{Red}{+0.01}&+0.11+0.11+0.18&-0.13-0.08+0.12&+0.20+0.15\textcolor{Red}{+0.05}&\textcolor{Red}{+0.32}\textcolor{Red}{+0.24}\textcolor{Red}{+0.00}&+0.04+0.02+0.37&-0.09-0.13+0.21&+0.19+0.15\textcolor{Red}{+0.04}&+0.15\textcolor{Red}{+0.27}+0.10&-0.17\textcolor{Red}{-0.22}+0.05& & & & & & & & & & & & & & & & & & & & &  \\ 
BX.GSR.CMCP.ZS&+0.14\textcolor{Red}{+0.24}+0.12&+0.01+0.14+0.42&-0.06-0.06+0.30&\textcolor{Red}{-0.31}\textcolor{Red}{-0.33}\textcolor{Red}{+0.00}&+0.13\textcolor{Red}{+0.33}+0.12&-0.08-0.02+0.23&+0.01+0.06+0.47&+0.14+0.14+0.11&+0.13\textcolor{Red}{+0.33}+0.12&+0.11+0.10+0.14&-0.09-0.11+0.22&-0.20-0.13\textcolor{Red}{+0.04}& & & & & & & & & & & & & & & & & & & &  \\ 
BX.KLT.DINV.WD.GD.ZS&-0.09\textcolor{Red}{-0.37}\textcolor{Red}{+0.00}&-0.06-0.17+0.20&+0.05-0.16+0.26&-0.02+0.10+0.66&-0.08\textcolor{Red}{-0.40}\textcolor{Red}{+0.00}&-0.01+0.05+0.54&-0.13+0.00+0.11&-0.17-0.07\textcolor{Red}{+0.03}&-0.08\textcolor{Red}{-0.40}\textcolor{Red}{+0.00}&+0.19-0.02+0.08&\textcolor{Red}{+0.55}+0.16\textcolor{Red}{+0.01}&-0.15-0.09+0.06&-0.07-0.17+0.33& & & & & & & & & & & & & & & & & & &  \\ 
EG.GDP.PUSE.KO.PP&-0.18\textcolor{Red}{-0.37}+0.05&-0.18\textcolor{Red}{-0.25}\textcolor{Red}{+0.05}&-0.07-0.06+0.27&+0.10+0.19+0.16&-0.17\textcolor{Red}{-0.35}+0.06&\textcolor{Red}{-0.22}\textcolor{Red}{-0.28}\textcolor{Red}{+0.02}&-0.10\textcolor{Red}{-0.23}+0.20&-0.15-0.19+0.09&-0.17\textcolor{Red}{-0.34}\textcolor{Red}{+0.05}&-0.10-0.02+0.19&+0.18+0.13+0.05&-0.05-0.08+0.35&-0.20\textcolor{Red}{-0.26}\textcolor{Red}{+0.04}&+0.11+0.16+0.17& & & & & & & & & & & & & & & & & &  \\ 
EG.GDP.PUSE.KO.PP.KD&-0.19\textcolor{Red}{-0.40}\textcolor{Red}{+0.03}&-0.20\textcolor{Red}{-0.27}\textcolor{Red}{+0.01}&-0.04-0.03+0.38&+0.17\textcolor{Red}{+0.22}+0.07&-0.20\textcolor{Red}{-0.38}\textcolor{Red}{+0.02}&\textcolor{Red}{-0.23}\textcolor{Red}{-0.30}\textcolor{Red}{+0.02}&-0.15\textcolor{Red}{-0.26}+0.07&-0.16-0.18+0.08&-0.20\textcolor{Red}{-0.37}\textcolor{Red}{+0.02}&-0.11-0.02+0.15&+0.20+0.20\textcolor{Red}{+0.05}&+0.02-0.02+0.43&\textcolor{Red}{-0.27}\textcolor{Red}{-0.32}\textcolor{Red}{+0.01}&+0.04+0.10+0.32&\textcolor{Red}{+0.94}\textcolor{Red}{+0.93}\textcolor{Red}{+0.00}& & & & & & & & & & & & & & & & &  \\ 
EG.USE.COMM.KT.OE&\textcolor{Red}{+0.79}\textcolor{Red}{+0.75}\textcolor{Red}{+0.00}&+0.13\textcolor{Red}{+0.32}+0.08&+0.05+0.12+0.27&-0.14\textcolor{Red}{-0.37}\textcolor{Red}{+0.00}&\textcolor{Red}{+0.66}\textcolor{Red}{+0.75}\textcolor{Red}{+0.00}&+0.00\textcolor{Red}{+0.26}+0.47&-0.02+0.16+0.48&-0.10-0.09+0.19&\textcolor{Red}{+0.66}\textcolor{Red}{+0.75}\textcolor{Red}{+0.00}&\textcolor{Red}{+0.32}\textcolor{Red}{+0.38}\textcolor{Red}{+0.01}&+0.03-0.16+0.29&+0.10\textcolor{Red}{+0.23}+0.17&+0.15\textcolor{Red}{+0.28}+0.10&-0.07\textcolor{Red}{-0.39}\textcolor{Red}{+0.00}&-0.15\textcolor{Red}{-0.30}+0.07&-0.16\textcolor{Red}{-0.36}\textcolor{Red}{+0.05}& & & & & & & & & & & & & & & &  \\ 
EG.USE.COMM.GD.PP.KD&+0.07\textcolor{Red}{+0.35}+0.19&+0.12\textcolor{Red}{+0.24}+0.11&-0.03-0.03+0.41&-0.12\textcolor{Red}{-0.26}+0.10&+0.09\textcolor{Red}{+0.36}+0.16&\textcolor{Red}{+0.42}\textcolor{Red}{+0.36}\textcolor{Red}{+0.00}&\textcolor{Red}{+0.32}\textcolor{Red}{+0.33}\textcolor{Red}{+0.01}&+0.11+0.19+0.15&+0.09\textcolor{Red}{+0.35}+0.17&+0.05+0.06+0.33&-0.12-0.18+0.10&-0.05+0.04+0.38&+0.03\textcolor{Red}{+0.24}+0.36&-0.05-0.01+0.35&\textcolor{Red}{-0.61}\textcolor{Red}{-0.87}\textcolor{Red}{+0.00}&\textcolor{Red}{-0.60}\textcolor{Red}{-0.86}\textcolor{Red}{+0.00}&+0.06\textcolor{Red}{+0.34}+0.22& & & & & & & & & & & & & & &  \\ 
EG.USE.ELEC.KH.PC&+0.10+0.01+0.15&+0.15+0.00+0.06&-0.15-0.05+0.06&+0.05\textcolor{Red}{-0.32}+0.23&+0.15+0.11+0.09&-0.03-0.04+0.43&\textcolor{Red}{-0.27}\textcolor{Red}{-0.32}\textcolor{Red}{+0.00}&\textcolor{Red}{-0.35}\textcolor{Red}{-0.48}\textcolor{Red}{+0.00}&+0.16+0.11+0.09&\textcolor{Red}{+0.46}\textcolor{Red}{+0.51}\textcolor{Red}{+0.00}&-0.02-0.13+0.48&+0.21\textcolor{Red}{+0.32}\textcolor{Red}{+0.04}&+0.05+0.08+0.31&\textcolor{Red}{+0.22}+0.12\textcolor{Red}{+0.03}&-0.13-0.06+0.10&-0.14-0.08+0.09&+0.12\textcolor{Red}{+0.40}+0.08&+0.09+0.15+0.18& & & & & & & & & & & & & &  \\ 
EN.ATM.CO2E.KD.GD&+0.07\textcolor{Red}{+0.24}+0.23&+0.13\textcolor{Red}{+0.24}+0.12&+0.08-0.01+0.24&-0.22-0.18\textcolor{Red}{+0.00}&+0.11\textcolor{Red}{+0.23}+0.15&\textcolor{Red}{+0.44}\textcolor{Red}{+0.35}\textcolor{Red}{+0.00}&\textcolor{Red}{+0.32}\textcolor{Red}{+0.43}\textcolor{Red}{+0.00}&+0.13\textcolor{Red}{+0.29}+0.12&+0.11\textcolor{Red}{+0.23}+0.19&-0.06-0.09+0.28&-0.07-0.07+0.26&+0.02+0.07+0.43&+0.03+0.09+0.39&-0.06-0.02+0.33&\textcolor{Red}{-0.47}\textcolor{Red}{-0.49}\textcolor{Red}{+0.00}&\textcolor{Red}{-0.46}\textcolor{Red}{-0.53}\textcolor{Red}{+0.00}&+0.07+0.19+0.20&\textcolor{Red}{+0.89}\textcolor{Red}{+0.63}\textcolor{Red}{+0.00}&-0.13-0.07+0.05& & & & & & & & & & & & &  \\ 
EN.ATM.CO2E.PC&\textcolor{Red}{+0.30}-0.02\textcolor{Red}{+0.01}&\textcolor{Red}{+0.36}+0.03\textcolor{Red}{+0.00}&-0.06-0.00+0.30&\textcolor{Red}{-0.25}\textcolor{Red}{-0.29}\textcolor{Red}{+0.00}&\textcolor{Red}{+0.33}+0.06\textcolor{Red}{+0.01}&-0.12-0.11+0.14&\textcolor{Red}{-0.24}\textcolor{Red}{-0.30}\textcolor{Red}{+0.01}&\textcolor{Red}{-0.34}\textcolor{Red}{-0.35}\textcolor{Red}{+0.00}&\textcolor{Red}{+0.33}+0.05\textcolor{Red}{+0.01}&\textcolor{Red}{+0.42}\textcolor{Red}{+0.51}\textcolor{Red}{+0.00}&+0.14-0.04+0.10&+0.17\textcolor{Red}{+0.29}+0.06&+0.03-0.03+0.38&\textcolor{Red}{+0.39}+0.13\textcolor{Red}{+0.01}&-0.15-0.08+0.10&-0.15-0.05+0.07&\textcolor{Red}{+0.28}\textcolor{Red}{+0.31}\textcolor{Red}{+0.02}&+0.10+0.17+0.19&\textcolor{Red}{+0.57}\textcolor{Red}{+0.84}\textcolor{Red}{+0.00}&+0.05+0.01+0.27& & & & & & & & & & & &  \\ 
EN.ATM.PM10.MC.M3&+0.03+0.17+0.33&-0.12-0.07+0.12&-0.05+0.02+0.34&-0.15-0.02+0.06&-0.05+0.13+0.36&\textcolor{Red}{+0.31}\textcolor{Red}{+0.30}\textcolor{Red}{+0.00}&\textcolor{Red}{+0.47}\textcolor{Red}{+0.43}\textcolor{Red}{+0.00}&\textcolor{Red}{+0.41}\textcolor{Red}{+0.41}\textcolor{Red}{+0.00}&-0.06+0.13+0.34&-0.12-0.16+0.16&-0.00+0.08+0.56&-0.10-0.14+0.19&\textcolor{Red}{+0.24}+0.09\textcolor{Red}{+0.01}&-0.13-0.10\textcolor{Red}{+0.04}&-0.01-0.08+0.49&-0.07-0.08+0.29&-0.00+0.08+0.56&+0.19+0.14+0.05&\textcolor{Red}{-0.27}\textcolor{Red}{-0.34}\textcolor{Red}{+0.00}&\textcolor{Red}{+0.24}\textcolor{Red}{+0.34}\textcolor{Red}{+0.04}&\textcolor{Red}{-0.22}\textcolor{Red}{-0.29}\textcolor{Red}{+0.01}& & & & & & & & & & &  \\ 
ER.H2O.INTR.K3&\textcolor{Red}{+0.63}\textcolor{Red}{+0.62}\textcolor{Red}{+0.00}&\textcolor{Red}{+0.27}\textcolor{Red}{+0.36}\textcolor{Red}{+0.03}&-0.15-0.18+0.06&-0.18\textcolor{Red}{-0.30}\textcolor{Red}{+0.00}&\textcolor{Red}{+0.86}\textcolor{Red}{+0.73}\textcolor{Red}{+0.00}&+0.14-0.00+0.11&+0.11-0.01+0.16&+0.04+0.04+0.36&\textcolor{Red}{+0.85}\textcolor{Red}{+0.73}\textcolor{Red}{+0.00}&+0.18\textcolor{Red}{+0.29}+0.05&-0.02-0.03+0.51&+0.11+0.16+0.18&+0.14\textcolor{Red}{+0.24}+0.12&-0.08\textcolor{Red}{-0.39}\textcolor{Red}{+0.01}&-0.05-0.09+0.37&-0.06-0.16+0.29&\textcolor{Red}{+0.57}\textcolor{Red}{+0.51}\textcolor{Red}{+0.00}&-0.01+0.10+0.57&+0.11+0.12+0.13&+0.01-0.02+0.35&+0.16+0.03+0.09&-0.13-0.17+0.08& & & & & & & & & &  \\ 
ER.H2O.INTR.PC&-0.07+0.11+0.28&+0.03\textcolor{Red}{+0.28}+0.27&\textcolor{Red}{-0.32}\textcolor{Red}{-0.38}\textcolor{Red}{+0.00}&+0.13\textcolor{Red}{-0.22}+0.11&-0.00\textcolor{Red}{+0.25}+0.68&-0.06-0.08+0.30&-0.16-0.12+0.07&-0.02+0.03+0.47&-0.00\textcolor{Red}{+0.26}+0.67&\textcolor{Red}{+0.26}+0.15\textcolor{Red}{+0.03}&-0.10-0.12+0.13&+0.08+0.14+0.21&+0.13+0.01+0.12&-0.00-0.17+0.71&-0.10+0.16+0.20&-0.12+0.08+0.12&-0.07+0.00+0.20&+0.07-0.10+0.18&\textcolor{Red}{+0.63}+0.05\textcolor{Red}{+0.00}&-0.10-0.15+0.15&+0.01-0.03+0.41&-0.19\textcolor{Red}{-0.31}\textcolor{Red}{+0.00}&+0.07\textcolor{Red}{+0.69}+0.18& & & & & & & & &  \\ 
FM.LBL.MQMY.GD.ZS&+0.04\textcolor{Red}{-0.27}+0.23&-0.04\textcolor{Red}{-0.29}+0.42&+0.03-0.04+0.34&-0.07+0.01+0.29&+0.03-0.19+0.26&\textcolor{Red}{-0.23}\textcolor{Red}{-0.37}\textcolor{Red}{+0.01}&\textcolor{Red}{-0.34}\textcolor{Red}{-0.47}\textcolor{Red}{+0.00}&\textcolor{Red}{-0.32}\textcolor{Red}{-0.31}\textcolor{Red}{+0.00}&+0.03-0.19+0.26&\textcolor{Red}{+0.34}\textcolor{Red}{+0.29}\textcolor{Red}{+0.00}&\textcolor{Red}{+0.53}\textcolor{Red}{+0.24}\textcolor{Red}{+0.00}&-0.03+0.06+0.40&-0.03-0.01+0.43&\textcolor{Red}{+0.80}\textcolor{Red}{+0.30}\textcolor{Red}{+0.00}&+0.12+0.17+0.14&+0.10+0.17+0.16&+0.11-0.04+0.09&-0.18-0.16\textcolor{Red}{+0.01}&\textcolor{Red}{+0.33}\textcolor{Red}{+0.43}\textcolor{Red}{+0.02}&-0.20\textcolor{Red}{-0.23}\textcolor{Red}{+0.00}&\textcolor{Red}{+0.54}\textcolor{Red}{+0.52}\textcolor{Red}{+0.00}&-0.16-0.11\textcolor{Red}{+0.03}&+0.02-0.09+0.31&+0.00-0.13+0.39& & & & & & & &  \\ 
FS.AST.PRVT.GD.ZS&+0.20-0.13+0.05&+0.12-0.14+0.14&-0.04+0.05+0.38&+0.06-0.06+0.26&+0.19-0.08+0.06&-0.21\textcolor{Red}{-0.24}\textcolor{Red}{+0.02}&\textcolor{Red}{-0.36}\textcolor{Red}{-0.39}\textcolor{Red}{+0.00}&\textcolor{Red}{-0.40}\textcolor{Red}{-0.33}\textcolor{Red}{+0.00}&+0.20-0.08\textcolor{Red}{+0.05}&\textcolor{Red}{+0.51}\textcolor{Red}{+0.53}\textcolor{Red}{+0.00}&+0.16+0.15+0.10&+0.11+0.15+0.17&-0.05-0.10+0.35&\textcolor{Red}{+0.25}\textcolor{Red}{+0.25}\textcolor{Red}{+0.04}&-0.03+0.09+0.42&-0.03+0.10+0.40&\textcolor{Red}{+0.31}+0.12\textcolor{Red}{+0.02}&-0.03-0.06+0.43&\textcolor{Red}{+0.69}\textcolor{Red}{+0.56}\textcolor{Red}{+0.00}&-0.20\textcolor{Red}{-0.26}\textcolor{Red}{+0.01}&\textcolor{Red}{+0.56}\textcolor{Red}{+0.65}\textcolor{Red}{+0.00}&\textcolor{Red}{-0.27}\textcolor{Red}{-0.30}\textcolor{Red}{+0.00}&+0.16+0.06+0.10&\textcolor{Red}{+0.36}-0.03\textcolor{Red}{+0.01}&\textcolor{Red}{+0.62}\textcolor{Red}{+0.80}\textcolor{Red}{+0.00}& & & & & & &  \\ 
IC.CRD.PRVT.ZS&\textcolor{Red}{+0.24}\textcolor{Red}{+0.34}\textcolor{Red}{+0.03}&\textcolor{Red}{+0.33}\textcolor{Red}{+0.22}\textcolor{Red}{+0.00}&+0.02+0.10+0.38&-0.08-0.08+0.27&\textcolor{Red}{+0.24}\textcolor{Red}{+0.35}\textcolor{Red}{+0.02}&-0.16-0.11+0.07&-0.21-0.14\textcolor{Red}{+0.03}&-0.18-0.10+0.06&\textcolor{Red}{+0.24}\textcolor{Red}{+0.35}\textcolor{Red}{+0.02}&\textcolor{Red}{+0.34}\textcolor{Red}{+0.46}\textcolor{Red}{+0.00}&-0.03-0.01+0.45&\textcolor{Red}{+0.24}\textcolor{Red}{+0.25}\textcolor{Red}{+0.03}&-0.06+0.02+0.31&-0.13-0.20\textcolor{Red}{+0.01}&+0.04+0.06+0.37&+0.03+0.07+0.36&+0.18\textcolor{Red}{+0.40}+0.06&-0.17-0.10\textcolor{Red}{+0.04}&\textcolor{Red}{+0.42}\textcolor{Red}{+0.31}\textcolor{Red}{+0.00}&\textcolor{Red}{-0.28}\textcolor{Red}{-0.27}\textcolor{Red}{+0.00}&\textcolor{Red}{+0.29}\textcolor{Red}{+0.29}\textcolor{Red}{+0.01}&\textcolor{Red}{-0.24}-0.21\textcolor{Red}{+0.00}&\textcolor{Red}{+0.25}\textcolor{Red}{+0.47}\textcolor{Red}{+0.02}&+0.18\textcolor{Red}{+0.29}+0.06&-0.00+0.10+0.54&\textcolor{Red}{+0.36}\textcolor{Red}{+0.38}\textcolor{Red}{+0.00}& & & & & &  \\ 
IC.EXP.DURS&-0.11+0.18+0.15&-0.07+0.03+0.25&+0.02-0.04+0.38&-0.02+0.08+0.46&-0.08+0.14+0.24&\textcolor{Red}{+0.41}\textcolor{Red}{+0.41}\textcolor{Red}{+0.00}&\textcolor{Red}{+0.44}\textcolor{Red}{+0.54}\textcolor{Red}{+0.00}&\textcolor{Red}{+0.38}\textcolor{Red}{+0.51}\textcolor{Red}{+0.00}&-0.08+0.13+0.25&\textcolor{Red}{-0.31}\textcolor{Red}{-0.39}\textcolor{Red}{+0.00}&-0.10-0.09+0.19&-0.12-0.12+0.15&+0.01-0.03+0.43&-0.15-0.19\textcolor{Red}{+0.01}&\textcolor{Red}{-0.36}\textcolor{Red}{-0.37}\textcolor{Red}{+0.00}&\textcolor{Red}{-0.33}\textcolor{Red}{-0.35}\textcolor{Red}{+0.00}&-0.12-0.05+0.10&\textcolor{Red}{+0.66}\textcolor{Red}{+0.38}\textcolor{Red}{+0.00}&\textcolor{Red}{-0.28}\textcolor{Red}{-0.51}\textcolor{Red}{+0.00}&\textcolor{Red}{+0.65}\textcolor{Red}{+0.45}\textcolor{Red}{+0.00}&\textcolor{Red}{-0.33}\textcolor{Red}{-0.55}\textcolor{Red}{+0.00}&\textcolor{Red}{+0.42}\textcolor{Red}{+0.36}\textcolor{Red}{+0.00}&-0.09-0.07+0.19&-0.10-0.12+0.12&\textcolor{Red}{-0.37}\textcolor{Red}{-0.57}\textcolor{Red}{+0.00}&\textcolor{Red}{-0.45}\textcolor{Red}{-0.64}\textcolor{Red}{+0.00}&\textcolor{Red}{-0.35}\textcolor{Red}{-0.40}\textcolor{Red}{+0.00}& & & & &  \\ 
IC.LGL.CRED.XQ&+0.11-0.11+0.17&+0.09-0.08+0.21&+0.21+0.15\textcolor{Red}{+0.03}&-0.15-0.07+0.09&-0.00-0.10+0.48&-0.09-0.13+0.20&\textcolor{Red}{-0.27}\textcolor{Red}{-0.32}\textcolor{Red}{+0.01}&\textcolor{Red}{-0.27}\textcolor{Red}{-0.29}\textcolor{Red}{+0.01}&+0.00-0.11+0.49&+0.11+0.07+0.16&+0.03+0.01+0.40&+0.14+0.12+0.10&+0.08+0.11+0.22&+0.09+0.19+0.25&-0.12-0.08+0.13&-0.10-0.06+0.19&+0.12+0.04+0.14&-0.10-0.01+0.21&\textcolor{Red}{+0.27}\textcolor{Red}{+0.33}\textcolor{Red}{+0.01}&-0.16-0.19+0.07&\textcolor{Red}{+0.28}\textcolor{Red}{+0.24}\textcolor{Red}{+0.00}&\textcolor{Red}{-0.25}\textcolor{Red}{-0.25}\textcolor{Red}{+0.02}&-0.05-0.03+0.31&+0.01-0.13+0.47&\textcolor{Red}{+0.23}\textcolor{Red}{+0.31}\textcolor{Red}{+0.01}&\textcolor{Red}{+0.35}\textcolor{Red}{+0.34}\textcolor{Red}{+0.00}&+0.15+0.17+0.09&\textcolor{Red}{-0.26}\textcolor{Red}{-0.33}\textcolor{Red}{+0.01}& & & &  \\ 
NE.RSB.GNFS.ZS&+0.07+0.17+0.23&+0.04+0.08+0.31&-0.18-0.19\textcolor{Red}{+0.05}&\textcolor{Red}{-0.27}\textcolor{Red}{-0.25}\textcolor{Red}{+0.01}&+0.09\textcolor{Red}{+0.23}+0.19&+0.01+0.14+0.50&+0.05+0.06+0.34&+0.03-0.05+0.40&+0.09\textcolor{Red}{+0.23}+0.20&+0.21\textcolor{Red}{+0.31}\textcolor{Red}{+0.03}&+0.11+0.02+0.15&-0.01+0.09+0.45&+0.11+0.09+0.15&\textcolor{Red}{+0.23}+0.04+0.06&-0.12-0.11+0.13&-0.11-0.10+0.15&+0.02\textcolor{Red}{+0.25}+0.43&\textcolor{Red}{+0.26}\textcolor{Red}{+0.23}\textcolor{Red}{+0.02}&\textcolor{Red}{+0.23}\textcolor{Red}{+0.38}\textcolor{Red}{+0.03}&\textcolor{Red}{+0.22}+0.09\textcolor{Red}{+0.03}&\textcolor{Red}{+0.44}\textcolor{Red}{+0.46}\textcolor{Red}{+0.00}&+0.00-0.08+0.49&+0.09+0.19+0.20&+0.07+0.14+0.25&+0.15+0.03+0.10&+0.09+0.16+0.20&+0.08+0.18+0.24&+0.02-0.15+0.42&-0.14-0.18+0.11& & &  \\ 
NE.TRD.GNFS.ZS&\textcolor{Red}{-0.31}\textcolor{Red}{-0.55}\textcolor{Red}{+0.00}&-0.17-0.20\textcolor{Red}{+0.04}&-0.10-0.13+0.20&-0.02+0.09+0.48&\textcolor{Red}{-0.30}\textcolor{Red}{-0.56}\textcolor{Red}{+0.00}&-0.05-0.02+0.29&-0.12-0.04+0.16&-0.17-0.15+0.05&\textcolor{Red}{-0.30}\textcolor{Red}{-0.55}\textcolor{Red}{+0.00}&+0.02-0.08+0.40&\textcolor{Red}{+0.29}+0.03\textcolor{Red}{+0.01}&-0.13-0.07+0.11&-0.03-0.09+0.41&\textcolor{Red}{+0.56}\textcolor{Red}{+0.64}\textcolor{Red}{+0.00}&+0.17+0.17+0.09&+0.10+0.11+0.18&\textcolor{Red}{-0.24}\textcolor{Red}{-0.44}\textcolor{Red}{+0.00}&+0.03+0.01+0.38&+0.05+0.12+0.28&+0.07+0.18+0.24&+0.18+0.09+0.06&\textcolor{Red}{-0.22}-0.17\textcolor{Red}{+0.02}&\textcolor{Red}{-0.30}\textcolor{Red}{-0.48}\textcolor{Red}{+0.00}&+0.08-0.15+0.19&\textcolor{Red}{+0.42}+0.15\textcolor{Red}{+0.00}&+0.10+0.10+0.17&\textcolor{Red}{-0.32}\textcolor{Red}{-0.36}\textcolor{Red}{+0.00}&-0.16-0.15+0.05&+0.02+0.05+0.46&+0.18+0.12+0.06& &  \\ 
Corruption.Index&-0.10\textcolor{Red}{+0.24}+0.16&\textcolor{Red}{-0.26}+0.00\textcolor{Red}{+0.01}&-0.05-0.04+0.34&+0.04+0.10+0.36&-0.12+0.19+0.13&\textcolor{Red}{+0.23}+0.21\textcolor{Red}{+0.02}&\textcolor{Red}{+0.38}\textcolor{Red}{+0.46}\textcolor{Red}{+0.00}&\textcolor{Red}{+0.44}\textcolor{Red}{+0.47}\textcolor{Red}{+0.00}&-0.13+0.19+0.15&\textcolor{Red}{-0.48}\textcolor{Red}{-0.37}\textcolor{Red}{+0.00}&-0.16-0.04+0.09&-0.13\textcolor{Red}{-0.22}+0.13&+0.04+0.12+0.36&\textcolor{Red}{-0.22}-0.21\textcolor{Red}{+0.03}&-0.19\textcolor{Red}{-0.31}\textcolor{Red}{+0.05}&-0.19\textcolor{Red}{-0.33}+0.05&-0.11-0.00+0.15&\textcolor{Red}{+0.22}\textcolor{Red}{+0.24}\textcolor{Red}{+0.02}&\textcolor{Red}{-0.59}\textcolor{Red}{-0.59}\textcolor{Red}{+0.00}&\textcolor{Red}{+0.31}\textcolor{Red}{+0.42}\textcolor{Red}{+0.00}&\textcolor{Red}{-0.60}\textcolor{Red}{-0.60}\textcolor{Red}{+0.00}&\textcolor{Red}{+0.41}\textcolor{Red}{+0.43}\textcolor{Red}{+0.00}&-0.09-0.04+0.21&-0.18-0.18+0.06&\textcolor{Red}{-0.42}\textcolor{Red}{-0.48}\textcolor{Red}{+0.00}&\textcolor{Red}{-0.59}\textcolor{Red}{-0.60}\textcolor{Red}{+0.00}&\textcolor{Red}{-0.51}\textcolor{Red}{-0.37}\textcolor{Red}{+0.00}&\textcolor{Red}{+0.56}\textcolor{Red}{+0.67}\textcolor{Red}{+0.00}&\textcolor{Red}{-0.27}\textcolor{Red}{-0.31}\textcolor{Red}{+0.00}&-0.15-0.12+0.08&-0.19-0.19\textcolor{Red}{+0.04}&  \\ \label{CORR_TABLE}\end{tabular}}
\end{sidewaystable}
\endgroup

\section{Full model}
   
After fitting linear model with all predictors we obtain following
significant cooefcients and their 95\% confidence intervals (the rest are
insignificant so not showing them):
\begin{Schunk}
\begin{Soutput}
                          Estimate   Std. Error   t value     Pr(>|t|)         2.5 %        97.5 %
(Intercept)          135.810155516 2.282406e+01  5.950307 2.453536e-07  89.988944394  1.816314e+02
AG.LND.ARBL.HA.PC    -26.930898892 1.108220e+01 -2.430103 1.865271e-02 -49.179351954 -4.682446e+00
AG.YLD.CREL.KG        -0.002641129 1.289932e-03 -2.047495 4.577688e-02  -0.005230776 -5.148258e-05
BX.KLT.DINV.WD.GD.ZS   0.237463260 8.487408e-02  2.797830 7.241420e-03   0.067071436  4.078551e-01
EG.USE.ELEC.KH.PC     -0.001328373 6.232705e-04 -2.131294 3.790587e-02  -0.002579640 -7.710501e-05
NE.TRD.GNFS.ZS        -0.115311034 4.515055e-02 -2.553923 1.368273e-02  -0.205954544 -2.466752e-02
\end{Soutput}
\end{Schunk}

The entire model (tested by $H_0$ all cooefcients are 0 against some of them are not
zero) is significant with F statistic p-value
2.58853727341801e-08
and Adjusted R-squared: 0.639273084259923.


\section{Veryfying model assumptions}

\begin{figure}[H]
\begin{center}
\includegraphics{figures/fig-DIAGN}
\caption{Diagnostic plots for Corruption.Index$\sim .$}
\end{center}
\end{figure}

Assumptions of multiple linear regression model:
\subsection{Quantitative response variable (Corruption.Index)} 
The response variable is quantitative.
\subsection{p-1 explanatory independent quantitative variables}
From the pairs Figure-\ref{PAIRS_PLOT} and correlation Table-\ref{CORR_TABLE} we can see
that some of the predictors are linearly dependant and they will be removed later by different selection
algorithms.
\subsection{Values of the predictors are deterministic}
We can safely assume that the values of predictors are deterministic because
they were collected by respected research organisation. 
\subsection{Corruption.Index are values of random variables satyfing
linear equations}
Looking at pairs Figure-\ref{PAIRS_PLOT} we can see that most of the predictor
variables influence response variable in linear way. Later I am going to try to
apply quadratic transformation of predictors to see if it can improve the model.
\subsection{Normality assumptions of errors}
  $\epsilon_i$ mutually independent random variables with mean 0 and
  variance $\sigma^2$. For testing purposes we assume that $\epsilon_i \sim
  N(0,\sigma^2)$ (mean 0 and constant variance). The distribution assumption is
  needed for testing purposes and for the least sqares estimates to be optimal.
  
Checking normality assumption for errors:
\begin{figure}[H]
\begin{center}
\includegraphics{figures/fig-QQ}
\caption{QQ plot}
\end{center}
\end{figure}
QQ plots depicts quite good linear trend so we can assume residuals have normal
distribution. The long-tailed distributions of errors could pose problems for
tools we used here but as we see there are no visible long tails on qqplot.
We can also run Shapiro-Wilk normality test on residuals which $H_0$ states that
data is normal. The p-value from the test is
0.23 so we cannot reject
$H_0$.
  
Residuals are estimators for the error term on the regression model so they have
to fullfil requirements imposed on the error term. 
\begin{figure}[H]
\advance\rightskip-0.5cm
\advance\leftskip-1cm
\begin{center}
\includegraphics{figures/fig-RES}
\caption{Residual plots and potential outliers}
\label{RES_PLOT}
\end{center}
\end{figure}
From residual plot $\hat{\epsilon} \sim \hat{Y}$ we can conclude that variance
for observations with $\hat{Y}<50$ is smaller then variance for cases with
$\hat{Y}>50$. Because I assumed independence of errors but it looks that errors
are not identically distributed we can try two approaches i.e. try to fit two
models for those two groups separately or use generalized linear model like
weighted least squares.
P-value of F test comparing those two variances equals:
0.008.

\begin{figure}[H]
\begin{center}
\advance\rightskip-0.5cm
\advance\leftskip-1cm
\includegraphics[width=8.5in,height=11in]{figures/fig-ST_RES_VARS.pdf}
\caption{Studentized residual plots for predictors}
\label{ST_RES_VARS_PLOT}
\end{center}
\end{figure}
 
We can check if there is linear pattern in the errors by fitting the regression
line to residuals against fitted values:
\begin{Schunk}
\begin{Soutput}
                     Estimate Std. Error   t value   Pr(>|t|)
(Intercept)        6.29350551  2.8065569 2.2424293 0.02766685
fitted(model.full) 0.02297039  0.0391972 0.5860212 0.55949073
\end{Soutput}
\end{Schunk}
The coefficient for the fitted values isn't significant so we can conclude
there is no liner relationship in the residuals.

Using partial regression and partial residual plots we can look for
transformations of the predictor variables that could be benefitial to the
model: 
\begin{figure}[H]
\begin{center}
\advance\rightskip-0.5cm
\advance\leftskip-1cm
\includegraphics[width=8.5in,height=11in]{figures/fig-PAR_REGR.pdf}
\caption{Partial regression plots}
\end{center}
\label{PAR_REGR_PLOT}
\end{figure}

\begin{figure}[H]
\begin{center}
\advance\rightskip-0.5cm
\advance\leftskip-0.8cm
\includegraphics[width=8.5in,height=11in]{figures/fig-PAR_RES.pdf}
\caption{Partial residual plots}
\end{center}
\label{PAR_RES_PLOT}
\end{figure}
  
  \subsection{Additive impact of each predictor variable on explained
  variable.}
  This means that there are no interection effects between predictor variables
  included in the model.
  \subsection{Collinearity}
  Columns of experiment matrix have to be
  alegbraically independent otherwise we might not find solution or the
  solution is not stable.\\

The first check is to look at correlation matrix Table-\ref{CORR_TABLE}. The
table marks all significant correlations in red font. The following pairs of
variables have correlation bigger than 0.8:
\begin{Schunk}
\begin{Soutput}
[1] "AG.SRF.TOTL.K2 AG.LND.TOTL.K2"          "ER.H2O.INTR.K3 AG.LND.TOTL.K2"         
[3] "ER.H2O.INTR.K3 AG.SRF.TOTL.K2"          "EG.GDP.PUSE.KO.PP.KD EG.GDP.PUSE.KO.PP"
[5] "EN.ATM.CO2E.KD.GD EG.USE.COMM.GD.PP.KD"
\end{Soutput}
\end{Schunk}

Checking coolinearity by computing
rank of the matrix:
\begin{Schunk}
\begin{Sinput}
> X<-model.matrix(model.full)
> ncol(X)
\end{Sinput}
\begin{Soutput}
[1] 32
\end{Soutput}
\begin{Sinput}
> as.integer(rankMatrix(X))
\end{Sinput}
\begin{Soutput}
[1] 32
\end{Soutput}
\end{Schunk}

We can also check if predictors are not collinear by checking that eigenvalues
of the $X'X$ are not close to zero. The convenient way is to use statistic
$\kappa=\sqrt{\frac{\lambda_l}{\lambda_p}}$ where $\lambda_l$ is the largest
eigenvalue and $\lambda_p$ are other lambda values. Values greater equal than 30
are considered as a problem  \cite{Faraway}. As we can see we observe very large
values of this statistic: 
\begin{Schunk}
\begin{Soutput}
 [1] 6.678691e+07 3.210068e+07 7.978557e+06 5.688988e+06 2.318929e+06 1.781263e+06 1.358323e+06
 [8] 9.400946e+05 6.765676e+05 5.585418e+05 5.268543e+05 5.005712e+05 3.706954e+05 3.510966e+05
[15] 2.801852e+05 2.742668e+05 2.050105e+05 1.746508e+05 1.406485e+05 1.049815e+05 9.558464e+04
[22] 5.262181e+04 3.943306e+04 1.909145e+04 9.664668e+03 1.698878e+03 7.600963e+02 1.026057e+02
[29] 5.196613e+01 2.020892e+01 7.122399e+00 1.000000e+00
\end{Soutput}
\end{Schunk}
We can also check for multiple collinearity by computing variance inflation
factor (VIF) and removing variables with VIF $\geq$ 10. The VIF is
$\frac{1}{1-R_i^2}$ where $R_i^2$ is multiple coefficient of determination for
the model $X_i \sim X_1 + ... + X_{i-1} + X_{i+1} + X_{p-1}$. The standard error
of coefficient $\beta_i$ is proportional to the $\sqrt{VIF_i}$
($SE_{\hat{\beta_i}}=\sigma\sqrt{VIF_i}\frac{1}{\sqrt{S_{x_ix_i}}}$
\cite{Faraway}). VIF values:
\begin{Schunk}
\begin{Sinput}
> vif(model.full)	
\end{Sinput}
\begin{Soutput}
      AG.LND.AGRI.K2    AG.LND.ARBL.HA.PC       AG.LND.ARBL.ZS       AG.LND.CROP.ZS 
           18.632807             6.706063             1.790606             2.416085 
      AG.LND.TOTL.K2       AG.PRD.CROP.XD       AG.PRD.FOOD.XD       AG.PRD.LVSK.XD 
         3853.433087            12.846249            22.438777             7.816655 
      AG.SRF.TOTL.K2       AG.YLD.CREL.KG       BM.GSR.INSF.ZS       BM.GSR.TRVL.ZS 
         3826.562138             2.435269             2.432500             2.241709 
      BX.GSR.CMCP.ZS BX.KLT.DINV.WD.GD.ZS    EG.GDP.PUSE.KO.PP EG.GDP.PUSE.KO.PP.KD 
            1.863568            11.206991            19.456279            15.394738 
   EG.USE.COMM.KT.OE EG.USE.COMM.GD.PP.KD    EG.USE.ELEC.KH.PC    EN.ATM.CO2E.KD.GD 
           12.309089            17.612909             8.322555            12.706842 
      EN.ATM.CO2E.PC    EN.ATM.PM10.MC.M3       ER.H2O.INTR.K3       ER.H2O.INTR.PC 
            6.394319             2.696709             8.827993             4.295018 
   FM.LBL.MQMY.GD.ZS    FS.AST.PRVT.GD.ZS       IC.CRD.PRVT.ZS          IC.EXP.DURS 
           15.285510             9.945347             2.353677             4.255378 
      IC.LGL.CRED.XQ       NE.RSB.GNFS.ZS       NE.TRD.GNFS.ZS 
            1.974724             1.878799             2.797101 
\end{Soutput}
\end{Schunk}
We see that we should remove predictors from the model because there exists
multicollinearity between them. We remove predictors one by one until all
remaining predictors in the model have VIF<10. When all predictors have VID
below 10 the model consists of following coefficients: 
\begin{Schunk}
\begin{Soutput}
                        Estimate  Std. Error t value Pr(>|t|)    
(Intercept)           1.1840e+02  2.0637e+01  5.7372 3.86e-07 ***
AG.LND.AGRI.K2        1.5983e-06  3.9521e-06  0.4044  0.68742    
AG.LND.ARBL.HA.PC    -1.2532e+01  7.9489e+00 -1.5766  0.12042    
AG.LND.ARBL.ZS       -1.7453e-01  1.3540e-01 -1.2890  0.20262    
AG.LND.CROP.ZS       -1.5702e-01  2.5053e-01 -0.6268  0.53332    
AG.PRD.CROP.XD       -1.3246e-01  1.2623e-01 -1.0493  0.29845    
AG.PRD.LVSK.XD        3.7008e-02  1.0989e-01  0.3368  0.73753    
AG.YLD.CREL.KG       -2.9888e-03  1.1981e-03 -2.4946  0.01553 *  
BM.GSR.INSF.ZS       -3.6618e-01  2.1425e-01 -1.7091  0.09287 .  
BM.GSR.TRVL.ZS        2.2159e-01  1.4753e-01  1.5020  0.13862    
BX.GSR.CMCP.ZS       -4.8897e-02  9.1271e-02 -0.5357  0.59423    
BX.KLT.DINV.WD.GD.ZS  1.1236e-01  4.3433e-02  2.5871  0.01226 *  
EG.GDP.PUSE.KO.PP    -1.0611e+00  4.9750e-01 -2.1329  0.03725 *  
EG.USE.COMM.KT.OE     2.6057e-06  9.1950e-06  0.2834  0.77791    
EG.USE.ELEC.KH.PC    -1.0097e-03  5.6903e-04 -1.7745  0.08132 .  
EN.ATM.CO2E.KD.GD     1.7046e+00  1.3800e+00  1.2353  0.22180    
EN.ATM.CO2E.PC       -1.2679e+00  6.4869e-01 -1.9546  0.05554 .  
EN.ATM.PM10.MC.M3     7.9668e-02  5.7451e-02  1.3867  0.17093    
ER.H2O.INTR.K3       -1.6113e-04  2.5301e-03 -0.0637  0.94944    
ER.H2O.INTR.PC        3.7347e-05  3.7404e-05  0.9985  0.32228    
FS.AST.PRVT.GD.ZS    -1.7201e-02  4.5527e-02 -0.3778  0.70697    
IC.CRD.PRVT.ZS       -1.0732e-01  5.9733e-02 -1.7967  0.07768 .  
IC.EXP.DURS           4.5480e-02  1.5607e-01  0.2914  0.77180    
IC.LGL.CRED.XQ       -1.5262e-02  8.1848e-01 -0.0186  0.98519    
NE.RSB.GNFS.ZS        3.6597e-02  1.3291e-01  0.2754  0.78404    
NE.TRD.GNFS.ZS       -1.0128e-01  4.4178e-02 -2.2925  0.02559 *  
---
Signif. codes:  0 '***' 0.001 '**' 0.01 '*' 0.05 '.' 0.1 ' ' 1
\end{Soutput}
\end{Schunk}

The resulting model has comparable adjusted
$R^2=$0.75 to the full model.
It doesn't explain variance worse than the full model (p-value of F-statistic
comparing both models:
0.41). The list of
coefficients included in the model selected by the VIF method doesn't contain simultaneously both predictors that correlation is bigger than 0.8 (VIF method caters for pairwise collinearity and additionaly for
multiple collinearity). The resulting model has still some coefficients which
are not significant.

\begin{figure}[H]
\begin{center}
\includegraphics{figures/fig-VIF_MODEL_DIAG_FIG}
\caption{Diagnostic plots for model selected by VIF method}
\end{center}
\end{figure}

Checking if it would be beneficial to add quadratic terms:
\begin{Schunk}
\begin{Soutput}
[1] FALSE
\end{Soutput}
\end{Schunk}

\subsection{$p \leq n$}
\begin{Schunk}
\begin{Sinput}
> length(all.predictor.names)+1
\end{Sinput}
\begin{Soutput}
[1] 32
\end{Soutput}
\begin{Sinput}
> nrow(kaggle.data)
\end{Sinput}
\begin{Soutput}
[1] 83
\end{Soutput}
\end{Schunk}
\subsection{is specification of the structural equation of the model correct?}
  
The divergence from the structural assumptions can be check by analyzing plots:
\begin{itemize}
  \item Residual plots Figure-\ref{RES_PLOT}, Figure-\ref{ST_RES_VARS_PLOT},
  \item Response varaible against each predictor variable Figure-\ref{PAIRS_PLOT}
  \item Partial regression Figure-\ref{PAR_REGR_PLOT}. This allowes us to
  visualize relationshop between response variable and specific predictor taking
  out the effects of other predictors. I cannot see any outstanding nonlinearity there
  which could have meant that we should change the functional
  contribution of the predictor into the model.
  \item Partial residuals Figure-\ref{PAR_RES_PLOT} is better for spotting
  nonlinearity. The plots show partial residual plots and two lines i.e. least
  sqares and nonparametric smooth which allows us to compare linear and
  curved aproximation of the data. We can see that all sub figures show the
  linear aproximation is good.
\end{itemize}

One of the possibilties to deal with the heteroscedasticity of variance in error
term is to transform response variable. One of the procedures is Box-Cox
transformation, that computes the likelihood of model given transformation of
the response variable:
\begin{flalign*}
& g_\lambda(y)= \begin{cases} \frac{y^\lambda-1}{\lambda} \text{   } \lambda \neq 0 \\
                              log(y) \text{   } \lambda = 0 \end{cases}
\end{flalign*}
\begin{figure}[H]
\begin{center}
\advance\rightskip-0.5cm
\advance\leftskip-1cm
\includegraphics{figures/fig-BOXCOX}
\caption{Box-Cox graph showing log-likelihood of data depending on different
values $\lambda$}
\end{center}
\end{figure}
It shows that we can stay with the not transformed response variable because
log-likelihood of the the data has maximum near $\lambda=1$. 
I am also trying to tranform predictor variables with polynomials for different
models in this project.
 
  \subsection{Dependence of errors}
  Because this is not a timeseries and it is difficult to order observations in
  any way I assume that there is no dependence of errors.
  \subsection{Occurance of outliers or influential observations}
  
Potential outliers were shown on the residual plots but we have to remember
that influential obvservations can draw regression line towards them and
influential observations can be undetected when looking only at outliers
(moreover influetial observations don't have to be outliers).\\
Leverages can be used to find potential influential observations. Using
heuristic rule that observation is potentially influential if $h_{ii}\geq \frac{2p}{n}$ we can try to identify them:
\begin{Schunk}
\begin{Soutput}
[1] "Austria"    "Brazil"     "Canada"     "Iceland"    "Luxembourg" "Russia"    
\end{Soutput}
\end{Schunk}
But we have to remember that $h_i$ depends only on X so even if the observation
is far away from the mean(in terms of Mahalanpbis distance) it can still fit
into the model. Therefore better tool to diagnose observation as influential is
Cook's distance.
The outliers for leverages can be also analized on half normal plot on which we
see that we shouldn't be worried about unusual leverage values in our data:
\begin{figure}[H]
\begin{center}
\includegraphics{figures/fig-HALF_NORMAL_LEVERAGES}
\caption{Half normal plot for the leverage values from the full model}
\end{center}
\end{figure}
Another test for outliers uses jacknife residuals and Bonferroni critical value.
The maximum jacknife residual is 3.27 for
Israel/West Bank and the Bonferroni critical value is
-3.66
so we conclude that it is not an outlier because jacknife residual is less than
absolute value of critical value.\\
To identify influential observations we can look at halfnormal plot of the
Cook's distances:
\begin{figure}[H]
\begin{center}
\includegraphics{figures/fig-HALF_NORMAL_COOK}
\caption{Half normal plot of the Cook statistics}
\end{center}
\end{figure}
  
  \subsection{All significant predictors are included}
  We can be sure that there are some predictors out there (and not included in
  the provided data set) that would be beneficial to explain the response
  variable. I am not going to look for them.

\section{Selection of variables}
In this section I will fit different models based basen of different methods for
selection of variables:



\subsection{Backward elimination procedure based on t-tests}
Select sub-model (starting from the full model) until all coefficients are
significant at $\alpha_{crit}=$0.2. In each step we remove least
significant predictor with the biggest p-value $>\alpha_{crit}$. We have to
remember the removing variable from the model doesn't mean that the variable
doesn't explain predicted variable. It means that all other variables already in
the model have the same information as removed variable. Sometimes for the
meaningfulness sake of the model it is better to retain variable even if its
coefficient is nonsignificant. 
The model constructed this way:
\begin{Schunk}
\begin{Soutput}
                        Estimate  Std. Error t value  Pr(>|t|)    
(Intercept)           1.2270e+02  1.2967e+01  9.4626 4.968e-14 ***
AG.LND.ARBL.HA.PC    -1.0192e+01  5.0617e+00 -2.0137  0.048005 *  
AG.LND.ARBL.ZS       -2.6730e-01  1.0722e-01 -2.4931  0.015104 *  
AG.PRD.CROP.XD       -1.4715e-01  9.8167e-02 -1.4990  0.138497    
AG.YLD.CREL.KG       -2.2646e-03  9.8107e-04 -2.3083  0.024030 *  
BM.GSR.TRVL.ZS        2.9820e-01  1.3472e-01  2.2135  0.030223 *  
BX.KLT.DINV.WD.GD.ZS  1.4586e-01  4.8594e-02  3.0017  0.003752 ** 
EG.GDP.PUSE.KO.PP    -1.2133e+00  4.1566e-01 -2.9190  0.004758 ** 
EG.USE.ELEC.KH.PC    -7.8406e-04  2.8641e-04 -2.7376  0.007893 ** 
EN.ATM.CO2E.KD.GD     1.7789e+00  9.6981e-01  1.8343  0.070988 .  
EN.ATM.CO2E.PC       -1.2229e+00  4.1895e-01 -2.9191  0.004757 ** 
EN.ATM.PM10.MC.M3     8.3564e-02  4.2777e-02  1.9535  0.054877 .  
FM.LBL.MQMY.GD.ZS    -7.1500e-02  3.7050e-02 -1.9298  0.057801 .  
IC.CRD.PRVT.ZS       -1.3469e-01  5.3555e-02 -2.5150  0.014273 *  
NE.TRD.GNFS.ZS       -1.1227e-01  3.5577e-02 -3.1559  0.002385 ** 
---
Signif. codes:  0 '***' 0.001 '**' 0.01 '*' 0.05 '.' 0.1 ' ' 1
\end{Soutput}
\end{Schunk}

\begin{figure}[H]
\begin{center}
\includegraphics{figures/fig-DIAGN_BACK_EL_T_TEST}
\caption{Diagnostic plots for model selected by backward elimination procedure
based on t-tests}
\end{center}
\end{figure}

The final model with all significant predictors is not significantly worse than
the full model because p value of F statistic comparing this model to the full
model equals
0.93 so we should use the smaller model. Also the selected model explains data better then constant model because p value of F statistics comparing this model to the
constant model equals 1.23e-14.
Adjusted R-squared: 0.68.

We see that it would be good to reomve Luxembourg because Cook's distance is bigger
than one. After removing Luxembourg:

\begin{Schunk}
\begin{Sinput}
> printCoefmat(model.backward.t.test.sum$coefficients)
\end{Sinput}
\begin{Soutput}
                        Estimate  Std. Error t value  Pr(>|t|)    
(Intercept)           1.2233e+02  1.3016e+01  9.3990 7.425e-14 ***
AG.LND.ARBL.HA.PC    -1.0477e+01  5.0911e+00 -2.0580  0.043487 *  
AG.LND.ARBL.ZS       -2.7383e-01  1.0789e-01 -2.5380  0.013479 *  
AG.PRD.CROP.XD       -1.4576e-01  9.8488e-02 -1.4800  0.143557    
AG.YLD.CREL.KG       -2.3049e-03  9.8553e-04 -2.3388  0.022339 *  
BM.GSR.TRVL.ZS        3.1812e-01  1.3764e-01  2.3112  0.023904 *  
BX.KLT.DINV.WD.GD.ZS  2.3876e-01  1.3128e-01  1.8187  0.073428 .  
EG.GDP.PUSE.KO.PP    -1.2423e+00  4.1868e-01 -2.9671  0.004164 ** 
EG.USE.ELEC.KH.PC    -8.3881e-04  2.9614e-04 -2.8325  0.006095 ** 
EN.ATM.CO2E.KD.GD     1.7941e+00  9.7302e-01  1.8439  0.069625 .  
EN.ATM.CO2E.PC       -1.1987e+00  4.2145e-01 -2.8441  0.005900 ** 
EN.ATM.PM10.MC.M3     8.3616e-02  4.2909e-02  1.9487  0.055524 .  
FM.LBL.MQMY.GD.ZS    -6.6861e-02  3.7660e-02 -1.7754  0.080376 .  
IC.CRD.PRVT.ZS       -1.2987e-01  5.4093e-02 -2.4009  0.019138 *  
NE.TRD.GNFS.ZS       -1.2466e-01  3.9211e-02 -3.1791  0.002237 ** 
---
Signif. codes:  0 '***' 0.001 '**' 0.01 '*' 0.05 '.' 0.1 ' ' 1
\end{Soutput}
\end{Schunk}

\begin{figure}[H]
\begin{center}
\includegraphics{figures/fig-DIAGN_BACK_EL_T_TEST_2}
\caption{Diagnostic plots for model selected by backward elimination procedure
based on t-tests after removing Luxembourg}
\end{center}
\end{figure}
Adjusted R-squared:
0.67. We see also now
that there is no need to remove any more data.

There is a benefit in adding quadratic term for EG.GDP.PUSE.KO.PP:
\begin{Schunk}
\begin{Soutput}
                          Estimate  Std. Error t value  Pr(>|t|)    
(Intercept)             1.4520e+02  1.5514e+01  9.3592 1.005e-13 ***
AG.LND.ARBL.HA.PC      -1.2474e+01  4.9671e+00 -2.5112 0.0144866 *  
AG.LND.ARBL.ZS         -3.2730e-01  1.0607e-01 -3.0856 0.0029673 ** 
AG.PRD.CROP.XD         -1.8056e-01  9.5856e-02 -1.8837 0.0640119 .  
AG.YLD.CREL.KG         -2.1606e-03  9.5078e-04 -2.2725 0.0263197 *  
BM.GSR.TRVL.ZS          3.7830e-01  1.3471e-01  2.8082 0.0065450 ** 
BX.KLT.DINV.WD.GD.ZS    3.3713e-01  1.3240e-01  2.5463 0.0132283 *  
EG.GDP.PUSE.KO.PP      -6.1343e+00  1.9974e+00 -3.0711 0.0030963 ** 
EG.USE.ELEC.KH.PC      -1.1618e-03  3.1306e-04 -3.7111 0.0004258 ***
EN.ATM.CO2E.KD.GD       1.0369e+00  9.8470e-01  1.0531 0.2961548    
EN.ATM.CO2E.PC         -9.8227e-01  4.1497e-01 -2.3671 0.0208693 *  
EN.ATM.PM10.MC.M3       6.0344e-02  4.2355e-02  1.4247 0.1589460    
FM.LBL.MQMY.GD.ZS      -5.4510e-02  3.6600e-02 -1.4893 0.1411593    
IC.CRD.PRVT.ZS         -9.1053e-02  5.4353e-02 -1.6752 0.0986244 .  
NE.TRD.GNFS.ZS         -1.2991e-01  3.7818e-02 -3.4351 0.0010287 ** 
I(EG.GDP.PUSE.KO.PP^2)  2.4597e-01  9.8364e-02  2.5006 0.0148894 *  
---
Signif. codes:  0 '***' 0.001 '**' 0.01 '*' 0.05 '.' 0.1 ' ' 1
\end{Soutput}
\end{Schunk}
Adjusted R-squared:
0.69. Comparing the
model with quadratic term to model without it results in significant difference
with p-value 0.015.

\begin{figure}[H]
\begin{center}
\includegraphics{figures/fig-DIAGN_BACK_EL_T_TEST_QUAD}
\caption{Diagnostic plots for model selected by backward elimination procedure
based on t-tests after removing Luxembourg
and with quadratic term on EG.GDP.PUSE.KO.PP}
\end{center}
\end{figure}

\subsection{Forward selection procedure based on t-tests}
Starting from the constant model select extended model iteratively as far as all
coefficients are significant at $\alpha=0.1$, the selected model:
\begin{Schunk}
\begin{Soutput}
                        Estimate  Std. Error t value  Pr(>|t|)    
(Intercept)          95.87541621  7.40806456 12.9420 < 2.2e-16 ***
EN.ATM.CO2E.PC       -1.24009647  0.35510551 -3.4922 0.0008068 ***
IC.EXP.DURS           0.24747231  0.10384817  2.3830 0.0197042 *  
IC.CRD.PRVT.ZS       -0.17788245  0.05226977 -3.4032 0.0010718 ** 
EG.USE.ELEC.KH.PC    -0.00089759  0.00028492 -3.1503 0.0023430 ** 
NE.TRD.GNFS.ZS       -0.07498509  0.03106866 -2.4135 0.0182402 *  
EG.GDP.PUSE.KO.PP.KD -1.09084464  0.45203438 -2.4132 0.0182559 *  
AG.LND.ARBL.ZS       -0.21287596  0.10853296 -1.9614 0.0535450 .  
---
Signif. codes:  0 '***' 0.001 '**' 0.01 '*' 0.05 '.' 0.1 ' ' 1
\end{Soutput}
\end{Schunk}

\begin{figure}[H]
\begin{center}
\includegraphics{figures/fig-DIAGN_FORWARD_STEP_T_TEST}
\caption{Diagnostic plots for model selected by forward selection procedure based
on t-tests}
\end{center}
\end{figure}
The selected model by the forward procedure is even smaller than the model
selected by the backwards procedure but is still insignificantly worse than the
full model because p value of F statistic comparing this model to the full
model equals
0.5. We cannot compare directly (using for example F-test) the models created by the forward and backward procedures because none of them is the subset of the other one. Also the selected model explains data better then
constant model because p value of F statistics comparing this model to the
constant model equals
8.25e-16

We see that it would be good to reomve Iceland because Cook's distance is bigger
than one. After removing Icleland:

\begin{Schunk}
\begin{Soutput}
                        Estimate  Std. Error t value  Pr(>|t|)    
(Intercept)          96.23855689  7.26302874 13.2505 < 2.2e-16 ***
EN.ATM.CO2E.PC       -0.60477410  0.46928872 -1.2887  0.201516    
IC.EXP.DURS           0.22894628  0.10219686  2.2402  0.028077 *  
IC.CRD.PRVT.ZS       -0.16972368  0.05138995 -3.3027  0.001477 ** 
EG.USE.ELEC.KH.PC    -0.00188657  0.00056401 -3.3449  0.001295 ** 
NE.TRD.GNFS.ZS       -0.07894455  0.03051418 -2.5871  0.011640 *  
EG.GDP.PUSE.KO.PP.KD -1.02322531  0.44431344 -2.3029  0.024095 *  
AG.LND.ARBL.ZS       -0.21884654  0.10641656 -2.0565  0.043260 *  
---
Signif. codes:  0 '***' 0.001 '**' 0.01 '*' 0.05 '.' 0.1 ' ' 1
\end{Soutput}
\end{Schunk}

\begin{figure}[H]
\begin{center}
\includegraphics{figures/fig-DIAGN_FORWARD_STEP_T_TEST_2}
\caption{Diagnostic plots for model selected by forward selection procedure based
on t-tests after removing Iceland}
\end{center}
\end{figure}
Adjusted R-squared:
0.67. We see also now
that there is no need to remove any more data.

Checking if it would be beneficial to add quadratic terms:
\begin{Schunk}
\begin{Soutput}
[1] FALSE
\end{Soutput}
\end{Schunk}

\subsection{Backward based on AIC criterion}
The selected model:
\begin{Schunk}
\begin{Soutput}
                        Estimate  Std. Error t value  Pr(>|t|)    
(Intercept)           1.2919e+02  1.3351e+01  9.6767 2.776e-14 ***
AG.LND.ARBL.HA.PC    -1.1068e+01  5.0564e+00 -2.1890  0.032141 *  
AG.LND.ARBL.ZS       -2.8288e-01  1.0695e-01 -2.6450  0.010200 *  
AG.PRD.CROP.XD       -1.8762e-01  9.9964e-02 -1.8769  0.064949 .  
AG.YLD.CREL.KG       -2.6056e-03  9.9531e-04 -2.6179  0.010960 *  
BM.GSR.INSF.ZS       -2.3762e-01  1.8647e-01 -1.2743  0.207022    
BM.GSR.TRVL.ZS        3.4640e-01  1.3973e-01  2.4790  0.015738 *  
BX.KLT.DINV.WD.GD.ZS  2.1704e-01  6.5934e-02  3.2918  0.001601 ** 
EG.GDP.PUSE.KO.PP    -1.1943e+00  4.1326e-01 -2.8899  0.005210 ** 
EG.USE.ELEC.KH.PC    -1.1414e-03  3.7093e-04 -3.0771  0.003042 ** 
EN.ATM.CO2E.KD.GD     1.8800e+00  9.6871e-01  1.9407  0.056569 .  
EN.ATM.CO2E.PC       -1.2294e+00  4.1858e-01 -2.9371  0.004558 ** 
EN.ATM.PM10.MC.M3     9.8251e-02  4.3246e-02  2.2719  0.026358 *  
FM.LBL.MQMY.GD.ZS    -1.1943e-01  5.8365e-02 -2.0463  0.044719 *  
FS.AST.PRVT.GD.ZS     7.4541e-02  5.7148e-02  1.3043  0.196645    
IC.CRD.PRVT.ZS       -1.3302e-01  5.4393e-02 -2.4455  0.017140 *  
NE.TRD.GNFS.ZS       -1.1798e-01  3.5523e-02 -3.3213  0.001463 ** 
---
Signif. codes:  0 '***' 0.001 '**' 0.01 '*' 0.05 '.' 0.1 ' ' 1
\end{Soutput}
\end{Schunk}
optimizes criterion AIC = -2max loglikelihood + 2p.

\begin{figure}[H]
\begin{center}
\includegraphics{figures/fig-DIAGN_BACK_AIC}
\caption{Diagnostic plots for model selected by Backward based on AIC criterion}
\end{center}
\end{figure}
The final model is not significantly worse than
the full model because p value of F statistic comparing this model to the full
model equals
0.97 so we should use the smaller model. Also the selected model explains data better then constant model because p value of F statistics comparing this model to the
constant model equals 
3.92e-14.

Adjusted R-squared:
0.69.

Checking if it would be beneficial to add quadratic terms:
\begin{Schunk}
\begin{Soutput}
[1] FALSE
\end{Soutput}
\end{Schunk}

\subsection{Forward based on AIC criterion}
The selected model:
\begin{Schunk}
\begin{Soutput}
                        Estimate  Std. Error t value  Pr(>|t|)    
(Intercept)           1.0335e+02  9.0541e+00 11.4144 < 2.2e-16 ***
EN.ATM.CO2E.PC       -5.1420e-01  4.9519e-01 -1.0384  0.302767    
IC.EXP.DURS           1.2524e-01  1.0981e-01  1.1405  0.258072    
IC.CRD.PRVT.ZS       -1.2962e-01  5.3329e-02 -2.4305  0.017721 *  
EG.USE.ELEC.KH.PC    -1.3578e-03  4.1751e-04 -3.2522  0.001784 ** 
NE.TRD.GNFS.ZS       -1.1061e-01  3.5546e-02 -3.1116  0.002720 ** 
EG.GDP.PUSE.KO.PP.KD -1.3995e+00  4.4945e-01 -3.1138  0.002702 ** 
AG.LND.ARBL.ZS       -1.7859e-01  1.1298e-01 -1.5808  0.118573    
AG.YLD.CREL.KG       -2.4845e-03  1.0252e-03 -2.4234  0.018045 *  
AG.LND.ARBL.HA.PC    -1.2453e+01  5.2381e+00 -2.3773  0.020258 *  
BM.GSR.TRVL.ZS        2.4588e-01  1.2025e-01  2.0448  0.044751 *  
BX.KLT.DINV.WD.GD.ZS  1.0403e-01  4.6455e-02  2.2394  0.028401 *  
ER.H2O.INTR.PC        4.4463e-05  3.0713e-05  1.4477  0.152290    
EN.ATM.PM10.MC.M3     5.9475e-02  4.2278e-02  1.4067  0.164059    
FM.LBL.MQMY.GD.ZS    -4.9525e-02  3.5719e-02 -1.3865  0.170111    
---
Signif. codes:  0 '***' 0.001 '**' 0.01 '*' 0.05 '.' 0.1 ' ' 1
\end{Soutput}
\end{Schunk}

\begin{figure}[H]
\begin{center}
\includegraphics{figures/fig-DIAGN_FWD_AIC}
\caption{Diagnostic plots for model selected by Forward based on AIC criterion}
\end{center}
\end{figure}

The final model is not significantly worse than
the full model because p value of F statistic comparing this model to the full
model equals
0.91 so we
 should use the smaller model. Also the selected model explains data better then constant model because p value of F statistics comparing this model to the constant model equals 
1.62e-14.

Adjusted R-squared:
0.68.

There is a benefit in adding quadratic term for EG.GDP.PUSE.KO.PP.KD:
\begin{Schunk}
\begin{Soutput}
                             Estimate  Std. Error t value  Pr(>|t|)    
(Intercept)                1.2335e+02  1.1485e+01 10.7396  3.29e-16 ***
EN.ATM.CO2E.PC            -5.4082e-01  4.7460e-01 -1.1395 0.2585434    
IC.EXP.DURS                2.6164e-02  1.1163e-01  0.2344 0.8154081    
IC.CRD.PRVT.ZS            -9.2974e-02  5.2928e-02 -1.7566 0.0835525 .  
EG.USE.ELEC.KH.PC         -1.4664e-03  4.0214e-04 -3.6465 0.0005205 ***
NE.TRD.GNFS.ZS            -1.1005e-01  3.4062e-02 -3.2308 0.0019146 ** 
EG.GDP.PUSE.KO.PP.KD      -6.1690e+00  1.8458e+00 -3.3421 0.0013634 ** 
AG.LND.ARBL.ZS            -2.4154e-01  1.1082e-01 -2.1796 0.0328041 *  
AG.YLD.CREL.KG            -2.5461e-03  9.8265e-04 -2.5910 0.0117322 *  
AG.LND.ARBL.HA.PC         -1.4515e+01  5.0788e+00 -2.8579 0.0056773 ** 
BM.GSR.TRVL.ZS             2.7525e-01  1.1575e-01  2.3779 0.0202704 *  
BX.KLT.DINV.WD.GD.ZS       1.0340e-01  4.4514e-02  2.3228 0.0232372 *  
ER.H2O.INTR.PC             3.5281e-05  2.9631e-05  1.1907 0.2379839    
EN.ATM.PM10.MC.M3          6.2571e-02  4.0528e-02  1.5439 0.1273261    
FM.LBL.MQMY.GD.ZS         -3.6719e-02  3.4563e-02 -1.0624 0.2918841    
I(EG.GDP.PUSE.KO.PP.KD^2)  2.5284e-01  9.5150e-02  2.6572 0.0098404 ** 
---
Signif. codes:  0 '***' 0.001 '**' 0.01 '*' 0.05 '.' 0.1 ' ' 1
\end{Soutput}
\end{Schunk}
Adjusted R-squared:
0.71. Comparing the
model with quadratic term to model without it results in significant difference
with p-value 0.01.

\begin{figure}[H]
\begin{center}
\includegraphics{figures/fig-DIAGN_FWD_AIC_3}
\caption{Diagnostic plots for model selected by Forward based on AIC criterion
and with quadratic term on EG.GDP.PUSE.KO.PP.KD}
\end{center}
\end{figure}

We see that it would be good to reomve Luxembourg because Cook's distance is bigger
than one. After removing Luxembourg:
		

\begin{Schunk}
\begin{Soutput}
                             Estimate  Std. Error t value  Pr(>|t|)    
(Intercept)                1.2282e+02  1.1537e+01 10.6459 5.762e-16 ***
EN.ATM.CO2E.PC            -4.8498e-01  4.8116e-01 -1.0079 0.3171659    
IC.EXP.DURS                3.7611e-02  1.1288e-01  0.3332 0.7400416    
IC.CRD.PRVT.ZS            -8.6694e-02  5.3669e-02 -1.6153 0.1110073    
EG.USE.ELEC.KH.PC         -1.5442e-03  4.1511e-04 -3.7199 0.0004138 ***
NE.TRD.GNFS.ZS            -1.2298e-01  3.7880e-02 -3.2467 0.0018364 ** 
EG.GDP.PUSE.KO.PP.KD      -6.3205e+00  1.8610e+00 -3.3964 0.0011606 ** 
AG.LND.ARBL.ZS            -2.4692e-01  1.1134e-01 -2.2177 0.0300215 *  
AG.YLD.CREL.KG            -2.5629e-03  9.8565e-04 -2.6003 0.0114841 *  
AG.LND.ARBL.HA.PC         -1.4813e+01  5.1072e+00 -2.9005 0.0050562 ** 
BM.GSR.TRVL.ZS             2.9628e-01  1.1909e-01  2.4878 0.0153860 *  
BX.KLT.DINV.WD.GD.ZS       1.9754e-01  1.2727e-01  1.5521 0.1254123    
ER.H2O.INTR.PC             3.7101e-05  2.9804e-05  1.2448 0.2175982    
EN.ATM.PM10.MC.M3          6.1596e-02  4.0661e-02  1.5149 0.1345820    
FM.LBL.MQMY.GD.ZS         -3.1198e-02  3.5358e-02 -0.8823 0.3807913    
I(EG.GDP.PUSE.KO.PP.KD^2)  2.6325e-01  9.6325e-02  2.7329 0.0080475 ** 
---
Signif. codes:  0 '***' 0.001 '**' 0.01 '*' 0.05 '.' 0.1 ' ' 1
\end{Soutput}
\end{Schunk}
		
\begin{figure}[H]
\begin{center}
\includegraphics{figures/fig-DIAGN_FWD_AIC_4}
\caption{Diagnostic plots for model selected by Forward based on AIC criterion
after adding quadratic term and removing Luxembourg}
\end{center}
\end{figure}
Adjusted R-squared:
0.69. We see also now
that there is no need to remove any more data.

\subsection{Backward based on BIC criterion}
The selected model:
\begin{Schunk}
\begin{Soutput}
                     Estimate  Std. Error t value  Pr(>|t|)    
(Intercept)       94.84170805  5.24636179 18.0776 < 2.2e-16 ***
AG.YLD.CREL.KG    -0.00212845  0.00099221 -2.1452 0.0351339 *  
EG.GDP.PUSE.KO.PP -1.49648844  0.37739030 -3.9654 0.0001644 ***
EG.USE.ELEC.KH.PC -0.00065991  0.00029538 -2.2341 0.0284172 *  
EN.ATM.CO2E.PC    -1.45618044  0.35431882 -4.1098 9.909e-05 ***
EN.ATM.PM10.MC.M3  0.11828346  0.04099303  2.8855 0.0050843 ** 
IC.CRD.PRVT.ZS    -0.12563130  0.04774311 -2.6314 0.0102920 *  
---
Signif. codes:  0 '***' 0.001 '**' 0.01 '*' 0.05 '.' 0.1 ' ' 1
\end{Soutput}
\end{Schunk}
is smaller than the one selected by AIC criterion because BIC more havily
penalizes bigger models BIC = -2max loglikelihood + plog(n) (where log(n) in our
case = 4.42).

\begin{figure}[H]
\begin{center}
\includegraphics{figures/fig-DIAGN_BACK_BIC}
\caption{Diagnostic plots for model selected by Backward based on BIC criterion}
\end{center}
\end{figure}

The final model is not significantly worse than
the full model because p value of F statistic comparing this model to the full
model equals
0.34 so we should use the smaller model. Also the selected model explains data better then constant model because p value of F statistics comparing this model to the
constant model equals 
1.56e-15.
Adjusted R-squared:
0.62.

We see that it would be good to reomve Iceland because Cook's distance is bigger
than one. After removing Icleland:

\begin{Schunk}
\begin{Sinput}
> printCoefmat(model.bckwrd.bic.sum$coefficients)
\end{Sinput}
\begin{Soutput}
                     Estimate  Std. Error t value  Pr(>|t|)    
(Intercept)       94.43161723  5.20782714 18.1326 < 2.2e-16 ***
AG.YLD.CREL.KG    -0.00214356  0.00098366 -2.1792 0.0324590 *  
EG.GDP.PUSE.KO.PP -1.41121993  0.37826030 -3.7308 0.0003688 ***
EG.USE.ELEC.KH.PC -0.00144336  0.00059050 -2.4443 0.0168624 *  
EN.ATM.CO2E.PC    -0.95117427  0.48231715 -1.9721 0.0522863 .  
EN.ATM.PM10.MC.M3  0.11112888  0.04090677  2.7166 0.0081838 ** 
IC.CRD.PRVT.ZS    -0.11694075  0.04767003 -2.4531 0.0164850 *  
---
Signif. codes:  0 '***' 0.001 '**' 0.01 '*' 0.05 '.' 0.1 ' ' 1
\end{Soutput}
\end{Schunk}

\begin{figure}[H]
\begin{center}
\includegraphics{figures/fig-DIAGN_BACK_BIC_2}
\caption{Diagnostic plots for model selected by Backward based on BIC criterion after removing Iceland}
\end{center}
\end{figure}
Adjusted R-squared:
0.62. We see also now
that there is no need to remove any more data.

Checking if it would be beneficial to add quadratic terms:
\begin{Schunk}
\begin{Soutput}
[1] FALSE
\end{Soutput}
\end{Schunk}

\subsection{Forward based on BIC criterion}
The selected model:
\begin{Schunk}
\begin{Soutput}
                        Estimate  Std. Error t value  Pr(>|t|)    
(Intercept)          91.56208614  7.20541551 12.7074 < 2.2e-16 ***
EN.ATM.CO2E.PC       -1.26113093  0.36153075 -3.4883 0.0008118 ***
IC.EXP.DURS           0.25315374  0.10573428  2.3942 0.0191202 *  
IC.CRD.PRVT.ZS       -0.18368520  0.05315447 -3.4557 0.0009015 ***
EG.USE.ELEC.KH.PC    -0.00080268  0.00028599 -2.8067 0.0063559 ** 
NE.TRD.GNFS.ZS       -0.07133965  0.03158858 -2.2584 0.0267895 *  
EG.GDP.PUSE.KO.PP.KD -1.03600146  0.45954177 -2.2544 0.0270501 *  
---
Signif. codes:  0 '***' 0.001 '**' 0.01 '*' 0.05 '.' 0.1 ' ' 1
\end{Soutput}
\end{Schunk}

\begin{figure}[H]
\begin{center}
\includegraphics{figures/fig-DIAGN_FWD_BIC}
\caption{Diagnostic plots for model selected by Forward based on BIC criterion}
\end{center}
\end{figure}

The final model is not significantly worse than
the full model because p value of F statistic comparing this model to the full
model equals
0.38 so we should use the smaller model. Also the selected model explains data better then constant model because p value of F statistics comparing this model to the
constant model equals 
9.81e-16.

We see that it would be good to reomve Iceland because Cook's distance is bigger
than one. After removing Icleland:

\begin{Schunk}
\begin{Sinput}
> printCoefmat(model.frwd.bic.sum$coefficients)
\end{Sinput}
\begin{Soutput}
                        Estimate  Std. Error t value  Pr(>|t|)    
(Intercept)          91.79738996  7.08225568 12.9616 < 2.2e-16 ***
EN.ATM.CO2E.PC       -0.64437873  0.47888144 -1.3456  0.182488    
IC.EXP.DURS           0.23530733  0.10432597  2.2555  0.027018 *  
IC.CRD.PRVT.ZS       -0.17591568  0.05239446 -3.3575  0.001238 ** 
EG.USE.ELEC.KH.PC    -0.00176106  0.00057265 -3.0753  0.002933 ** 
NE.TRD.GNFS.ZS       -0.07508760  0.03110525 -2.4140  0.018219 *  
EG.GDP.PUSE.KO.PP.KD -0.96880297  0.45297230 -2.1388  0.035710 *  
---
Signif. codes:  0 '***' 0.001 '**' 0.01 '*' 0.05 '.' 0.1 ' ' 1
\end{Soutput}
\end{Schunk}

\begin{figure}[H]
\begin{center}
\includegraphics{figures/fig-DIAGN_FWD_BIC_2}
\caption{Diagnostic plots for model selected by Forward based on BIC criterion after removing Iceland}
\end{center}
\end{figure}
Adjusted R-squared:
0.63. We see also now
that there is no need to remove any more data.

There is a benefit in adding quadratic term for EN.ATM.CO2E.P:
\begin{Schunk}
\begin{Soutput}
                        Estimate  Std. Error t value  Pr(>|t|)    
(Intercept)          96.82967979  7.38886576 13.1048 < 2.2e-16 ***
EN.ATM.CO2E.PC       -2.11723759  0.87467105 -2.4206  0.017948 *  
IC.EXP.DURS           0.21949283  0.10261681  2.1390  0.035739 *  
IC.CRD.PRVT.ZS       -0.16982687  0.05147271 -3.2994  0.001492 ** 
EG.USE.ELEC.KH.PC    -0.00179809  0.00056189 -3.2001  0.002024 ** 
NE.TRD.GNFS.ZS       -0.08221327  0.03071251 -2.6769  0.009147 ** 
EG.GDP.PUSE.KO.PP.KD -1.07122234  0.44717496 -2.3955  0.019126 *  
I(EN.ATM.CO2E.PC^2)   0.08028523  0.04022281  1.9960  0.049613 *  
---
Signif. codes:  0 '***' 0.001 '**' 0.01 '*' 0.05 '.' 0.1 ' ' 1
\end{Soutput}
\end{Schunk}
Adjusted R-squared:
0.64. Comparing the
model with quadratic term to model without it results in significant difference
with p-value 0.05.

\begin{figure}[H]
\begin{center}
\includegraphics{figures/fig-DIAGN_FWD_BIC_QUAD}
\caption{Diagnostic plots for model selected by Forward based on BIC criterion after removing Iceland
and with quadratic term on EN.ATM.CO2E.P}
\end{center}
\end{figure}


\subsection{Based on Adjusted R2 criterion}
To select best model using $R^2$ we have to use adjusted $R^2$ becuase we have
to take into considaration number of parameters in the model and not only how
well model fits the data.

\begin{figure}[H]
\begin{center}
\includegraphics{figures/fig-FIG_ADJ_R2_PARAMS}
\caption{Adjusted $R^2$ against number of model parameters in selected best
model.}
\end{center}
\end{figure}

\begin{figure}[H]
\begin{center}
\advance\rightskip-0.5cm
\advance\leftskip-1cm
\includegraphics[width=8.5in,height=11in]{figures/fig-ADJ_R2_VIS.pdf}
\caption{Finding best model using adjusted $R^2$ critierion.}
\end{center}
\end{figure}

Selected model with maximum adjusted $R^2$:
\begin{Schunk}
\begin{Soutput}
                        Estimate  Std. Error t value  Pr(>|t|)    
(Intercept)           1.3673e+02  1.4529e+01  9.4112  1.09e-13 ***
AG.LND.ARBL.HA.PC    -1.4650e+01  5.6392e+00 -2.5980 0.0116252 *  
AG.LND.ARBL.ZS       -2.4089e-01  1.1011e-01 -2.1876 0.0323548 *  
AG.LND.CROP.ZS       -2.3289e-01  2.0662e-01 -1.1272 0.2638836    
AG.PRD.CROP.XD       -2.3006e-01  1.0336e-01 -2.2257 0.0295625 *  
AG.SRF.TOTL.K2        7.1802e-07  6.5401e-07  1.0979 0.2763765    
AG.YLD.CREL.KG       -2.9829e-03  1.0247e-03 -2.9111 0.0049512 ** 
BM.GSR.INSF.ZS       -3.0973e-01  1.9282e-01 -1.6063 0.1131260    
BM.GSR.TRVL.ZS        3.5855e-01  1.3951e-01  2.5701 0.0125057 *  
BX.KLT.DINV.WD.GD.ZS  2.4313e-01  6.8819e-02  3.5329 0.0007687 ***
EG.GDP.PUSE.KO.PP    -1.1890e+00  4.1210e-01 -2.8853 0.0053244 ** 
EG.USE.ELEC.KH.PC    -1.0875e-03  3.7168e-04 -2.9260 0.0047476 ** 
EN.ATM.CO2E.KD.GD     1.7543e+00  9.6905e-01  1.8103 0.0749426 .  
EN.ATM.CO2E.PC       -1.4046e+00  4.3259e-01 -3.2470 0.0018581 ** 
EN.ATM.PM10.MC.M3     1.0193e-01  4.3583e-02  2.3387 0.0224881 *  
FM.LBL.MQMY.GD.ZS    -1.3850e-01  5.9977e-02 -2.3093 0.0241672 *  
FS.AST.PRVT.GD.ZS     9.0793e-02  5.9530e-02  1.5252 0.1321457    
IC.CRD.PRVT.ZS       -1.3711e-01  5.4743e-02 -2.5046 0.0148185 *  
NE.TRD.GNFS.ZS       -1.1015e-01  3.7473e-02 -2.9395 0.0045685 ** 
---
Signif. codes:  0 '***' 0.001 '**' 0.01 '*' 0.05 '.' 0.1 ' ' 1
\end{Soutput}
\end{Schunk}

\begin{figure}[H]
\begin{center}
\includegraphics{figures/fig-DIAGN_ADJ_R2_MODEL}
\caption{Diagnostic plots for model selected by adjusted $R^2$ criterion}
\end{center}
\end{figure}
Adjusted R-squared:
0.69.

There is a benefit in adding quadratic term for EG.GDP.PUSE.KO.PP:
\begin{Schunk}
\begin{Soutput}
                          Estimate  Std. Error t value  Pr(>|t|)    
(Intercept)             1.5496e+02  1.6873e+01  9.1836  3.12e-13 ***
AG.LND.ARBL.HA.PC      -1.5900e+01  5.5469e+00 -2.8665 0.0056358 ** 
AG.LND.ARBL.ZS         -2.7411e-01  1.0890e-01 -2.5172 0.0143870 *  
AG.LND.CROP.ZS         -2.8110e-01  2.0338e-01 -1.3822 0.1717967    
AG.PRD.CROP.XD         -2.5661e-01  1.0189e-01 -2.5184 0.0143417 *  
AG.SRF.TOTL.K2          6.6411e-07  6.3978e-07  1.0380 0.3032235    
AG.YLD.CREL.KG         -2.8471e-03  1.0038e-03 -2.8364 0.0061290 ** 
BM.GSR.INSF.ZS         -2.6678e-01  1.8968e-01 -1.4065 0.1644875    
BM.GSR.TRVL.ZS          3.8708e-01  1.3709e-01  2.8235 0.0063518 ** 
BX.KLT.DINV.WD.GD.ZS    2.3983e-01  6.7282e-02  3.5645 0.0007019 ***
EG.GDP.PUSE.KO.PP      -4.9478e+00  1.9224e+00 -2.5737 0.0124252 *  
EG.USE.ELEC.KH.PC      -1.2579e-03  3.7313e-04 -3.3713 0.0012824 ** 
EN.ATM.CO2E.KD.GD       1.1155e+00  9.9953e-01  1.1160 0.2686415    
EN.ATM.CO2E.PC         -1.2659e+00  4.2846e-01 -2.9545 0.0043991 ** 
EN.ATM.PM10.MC.M3       8.1030e-02  4.3860e-02  1.8475 0.0693738 .  
FM.LBL.MQMY.GD.ZS      -1.3396e-01  5.8664e-02 -2.2836 0.0257774 *  
FS.AST.PRVT.GD.ZS       8.9576e-02  5.8186e-02  1.5395 0.1286934    
IC.CRD.PRVT.ZS         -1.1485e-01  5.4650e-02 -2.1016 0.0395952 *  
NE.TRD.GNFS.ZS         -1.0557e-01  3.6696e-02 -2.8770 0.0054739 ** 
I(EG.GDP.PUSE.KO.PP^2)  1.8998e-01  9.5006e-02  1.9996 0.0498577 *  
---
Signif. codes:  0 '***' 0.001 '**' 0.01 '*' 0.05 '.' 0.1 ' ' 1
\end{Soutput}
\end{Schunk}
Adjusted R-squared:
0.7. Comparing the
model with quadratic term to model without it results in significant difference
with p-value 0.05.

\begin{figure}[H]
\begin{center}
\includegraphics{figures/fig-DIAGN_ADJ_R2_MODEL_CHECK_QUAD}
\caption{Diagnostic plots for model selected by adjusted $R^2$ criterion
and with quadratic term on EG.GDP.PUSE.KO.PP}
\end{center}
\end{figure}

We see that it would be good to reomve Luxembourg because Cook's distance is bigger
than one. After removing Luxembourg:
		

\begin{Schunk}
\begin{Soutput}
                          Estimate  Std. Error t value  Pr(>|t|)    
(Intercept)             1.5606e+02  1.6991e+01  9.1847 3.589e-13 ***
AG.LND.ARBL.HA.PC      -1.6057e+01  5.5692e+00 -2.8831  0.005406 ** 
AG.LND.ARBL.ZS         -2.8621e-01  1.1041e-01 -2.5923  0.011875 *  
AG.LND.CROP.ZS         -2.8151e-01  2.0406e-01 -1.3796  0.172679    
AG.PRD.CROP.XD         -2.4866e-01  1.0277e-01 -2.4197  0.018480 *  
AG.SRF.TOTL.K2          6.0943e-07  6.4591e-07  0.9435  0.349079    
AG.YLD.CREL.KG         -2.7760e-03  1.0115e-03 -2.7446  0.007915 ** 
BM.GSR.INSF.ZS         -2.0882e-01  2.0494e-01 -1.0189  0.312196    
BM.GSR.TRVL.ZS          4.0353e-01  1.3924e-01  2.8982  0.005182 ** 
BX.KLT.DINV.WD.GD.ZS    3.3210e-01  1.3861e-01  2.3959  0.019612 *  
EG.GDP.PUSE.KO.PP      -5.6541e+00  2.1399e+00 -2.6422  0.010413 *  
EG.USE.ELEC.KH.PC      -1.2730e-03  3.7490e-04 -3.3956  0.001200 ** 
EN.ATM.CO2E.KD.GD       1.0075e+00  1.0128e+00  0.9947  0.323753    
EN.ATM.CO2E.PC         -1.2048e+00  4.3729e-01 -2.7553  0.007689 ** 
EN.ATM.PM10.MC.M3       7.3750e-02  4.5031e-02  1.6378  0.106535    
FM.LBL.MQMY.GD.ZS      -1.1505e-01  6.3879e-02 -1.8010  0.076561 .  
FS.AST.PRVT.GD.ZS       7.1091e-02  6.3217e-02  1.1246  0.265112    
IC.CRD.PRVT.ZS         -1.0506e-01  5.6317e-02 -1.8655  0.066848 .  
NE.TRD.GNFS.ZS         -1.1873e-01  4.0661e-02 -2.9199  0.004876 ** 
I(EG.GDP.PUSE.KO.PP^2)  2.2378e-01  1.0514e-01  2.1285  0.037278 *  
---
Signif. codes:  0 '***' 0.001 '**' 0.01 '*' 0.05 '.' 0.1 ' ' 1
\end{Soutput}
\end{Schunk}
		
\begin{figure}[H]
\begin{center}
\includegraphics{figures/fig-DIAGN_ADJ_R2_MODEL_CHECK_QUAD_2}
\caption{Diagnostic plots for model selected by adjusted $R^2$ criterion
and with quadratic term on EG.GDP.PUSE.KO.PP and removing Luxembourg}
\end{center}
\end{figure}
Adjusted R-squared:
0.69. We see also now
that there is no need to remove any more data.

\subsection{Based on $C_p$ criterion}
Mallow's $C_p$ criterion selects model that should predict well i.e. it tries to
minimize average mean sqare error. 

\begin{figure}[H]
\begin{center}
\advance\rightskip-0.5cm
\advance\leftskip-1cm
\includegraphics[width=8.5in,height=11in]{figures/fig-CP_VIS.pdf}
\caption{Finding best model using $C_p$ critierion.}
\end{center}
\end{figure}

Checking $C_p$ with respect to number of parameters (A model with good fit
should have $C_p$ around or less then p):
\begin{figure}[H]
\begin{center}
\includegraphics{figures/fig-FIG_CP_VS_PARAMS}
\caption{$C_p$ against number of model parameters.}
\end{center}
\end{figure}
We see that the model with 11 predictors (choosen by $C_p$ criterion) meets
goodnes of fit requiremnt.

Selected model with minimum $C_p$:
\begin{Schunk}
\begin{Soutput}
                        Estimate  Std. Error t value  Pr(>|t|)    
(Intercept)           1.0604e+02  7.0884e+00 14.9590 < 2.2e-16 ***
AG.LND.ARBL.HA.PC    -1.5073e+01  4.7113e+00 -3.1992  0.002061 ** 
AG.YLD.CREL.KG       -3.0971e-03  9.7807e-04 -3.1665  0.002274 ** 
BM.GSR.TRVL.ZS        2.2644e-01  1.2011e-01  1.8852  0.063493 .  
BX.KLT.DINV.WD.GD.ZS  1.1227e-01  4.5017e-02  2.4939  0.014967 *  
EG.GDP.PUSE.KO.PP.KD -1.4667e+00  4.1939e-01 -3.4973  0.000815 ***
EG.USE.ELEC.KH.PC    -1.5972e-03  3.3762e-04 -4.7308 1.105e-05 ***
EN.ATM.PM10.MC.M3     8.5535e-02  3.9993e-02  2.1387  0.035898 *  
ER.H2O.INTR.PC        7.1739e-05  2.4657e-05  2.9095  0.004831 ** 
FM.LBL.MQMY.GD.ZS    -6.9397e-02  3.3049e-02 -2.0998  0.039299 *  
IC.CRD.PRVT.ZS       -1.3999e-01  5.2572e-02 -2.6628  0.009581 ** 
NE.TRD.GNFS.ZS       -1.1717e-01  3.4629e-02 -3.3836  0.001168 ** 
---
Signif. codes:  0 '***' 0.001 '**' 0.01 '*' 0.05 '.' 0.1 ' ' 1
\end{Soutput}
\end{Schunk}

\begin{figure}[H]
\begin{center}
\includegraphics{figures/fig-DIAGN_CP_MODEL}
\caption{Diagnostic plots for model selected by $C_p$ criterion}
\end{center}
\end{figure}
Adjusted R-squared:
0.67.

There is a benefit in adding quadratic term for EG.GDP.PUSE.KO.PP.KD:
\begin{Schunk}
\begin{Soutput}
                             Estimate  Std. Error t value  Pr(>|t|)    
(Intercept)                1.1831e+02  8.4215e+00 14.0488 < 2.2e-16 ***
AG.LND.ARBL.HA.PC         -1.6609e+01  4.5878e+00 -3.6202 0.0005526 ***
AG.YLD.CREL.KG            -3.0733e-03  9.4388e-04 -3.2561 0.0017427 ** 
BM.GSR.TRVL.ZS             2.4041e-01  1.1604e-01  2.0717 0.0419815 *  
BX.KLT.DINV.WD.GD.ZS       9.9778e-02  4.3727e-02  2.2818 0.0255425 *  
EG.GDP.PUSE.KO.PP.KD      -5.5264e+00  1.6742e+00 -3.3010 0.0015182 ** 
EG.USE.ELEC.KH.PC         -1.6939e-03  3.2809e-04 -5.1631 2.179e-06 ***
EN.ATM.PM10.MC.M3          8.0531e-02  3.8645e-02  2.0839 0.0408234 *  
ER.H2O.INTR.PC             6.7411e-05  2.3857e-05  2.8257 0.0061436 ** 
FM.LBL.MQMY.GD.ZS         -4.8775e-02  3.2942e-02 -1.4806 0.1431917    
IC.CRD.PRVT.ZS            -9.8790e-02  5.3343e-02 -1.8520 0.0682442 .  
NE.TRD.GNFS.ZS            -1.1040e-01  3.3526e-02 -3.2930 0.0015563 ** 
I(EG.GDP.PUSE.KO.PP.KD^2)  2.2271e-01  8.9117e-02  2.4990 0.0148018 *  
---
Signif. codes:  0 '***' 0.001 '**' 0.01 '*' 0.05 '.' 0.1 ' ' 1
\end{Soutput}
\end{Schunk}
Adjusted R-squared:
0.7. Comparing the
model with quadratic term to model without it results in significant difference
with p-value 0.015.

\begin{figure}[H]
\begin{center}
\includegraphics{figures/fig-DIAGN_CP_MODEL_CHECK_QUAD}
\caption{Diagnostic plots for model selected by $C_p$ criterion
and with quadratic term on EG.GDP.PUSE.KO.PP.KD}
\end{center}
\end{figure}

We see that it would be good to reomve Luxembourg because Cook's distance is bigger
than one. After removing Luxembourg:
		

\begin{Schunk}
\begin{Soutput}
                          Estimate  Std. Error t value  Pr(>|t|)    
(Intercept)             1.5606e+02  1.6991e+01  9.1847 3.589e-13 ***
AG.LND.ARBL.HA.PC      -1.6057e+01  5.5692e+00 -2.8831  0.005406 ** 
AG.LND.ARBL.ZS         -2.8621e-01  1.1041e-01 -2.5923  0.011875 *  
AG.LND.CROP.ZS         -2.8151e-01  2.0406e-01 -1.3796  0.172679    
AG.PRD.CROP.XD         -2.4866e-01  1.0277e-01 -2.4197  0.018480 *  
AG.SRF.TOTL.K2          6.0943e-07  6.4591e-07  0.9435  0.349079    
AG.YLD.CREL.KG         -2.7760e-03  1.0115e-03 -2.7446  0.007915 ** 
BM.GSR.INSF.ZS         -2.0882e-01  2.0494e-01 -1.0189  0.312196    
BM.GSR.TRVL.ZS          4.0353e-01  1.3924e-01  2.8982  0.005182 ** 
BX.KLT.DINV.WD.GD.ZS    3.3210e-01  1.3861e-01  2.3959  0.019612 *  
EG.GDP.PUSE.KO.PP      -5.6541e+00  2.1399e+00 -2.6422  0.010413 *  
EG.USE.ELEC.KH.PC      -1.2730e-03  3.7490e-04 -3.3956  0.001200 ** 
EN.ATM.CO2E.KD.GD       1.0075e+00  1.0128e+00  0.9947  0.323753    
EN.ATM.CO2E.PC         -1.2048e+00  4.3729e-01 -2.7553  0.007689 ** 
EN.ATM.PM10.MC.M3       7.3750e-02  4.5031e-02  1.6378  0.106535    
FM.LBL.MQMY.GD.ZS      -1.1505e-01  6.3879e-02 -1.8010  0.076561 .  
FS.AST.PRVT.GD.ZS       7.1091e-02  6.3217e-02  1.1246  0.265112    
IC.CRD.PRVT.ZS         -1.0506e-01  5.6317e-02 -1.8655  0.066848 .  
NE.TRD.GNFS.ZS         -1.1873e-01  4.0661e-02 -2.9199  0.004876 ** 
I(EG.GDP.PUSE.KO.PP^2)  2.2378e-01  1.0514e-01  2.1285  0.037278 *  
---
Signif. codes:  0 '***' 0.001 '**' 0.01 '*' 0.05 '.' 0.1 ' ' 1
\end{Soutput}
\end{Schunk}
		
\begin{figure}[H]
\begin{center}
\includegraphics{figures/fig-DIAGN_CP_MODEL_CHECK_QUAD_2}
\caption{Diagnostic plots for model selected by $C_p$ criterion
and with quadratic term on EG.GDP.PUSE.KO.PP.KD and removed Luxembourg}
\end{center}
\end{figure}
Adjusted R-squared:
0.69. We see also now
that there is no need to remove any more data.

\section{Shrinkage methods}
Methods in this section can be used to automatically select the predictors. 

\subsection{Ridge regression}
To choose biased estimate of the $\beta$ we can apply ridge regression which
deals with collnerity of the predictors.


Choose the best value of penalty parameter  using genralized crossvalidation:
\begin{Schunk}
\begin{Soutput}
modified HKB estimator is 0.1465822 
modified L-W estimator is 13.65141 
smallest value of GCV  at 39.5 
\end{Soutput}
\end{Schunk}

\begin{figure}[H]
\begin{center}
\advance\rightskip-0.5cm
\advance\leftskip-1cm
\includegraphics[width=8.5in,height=11in]{figures/fig-RIDGE.pdf}
\caption{fitted values of coefficients (bi) as a function of parameter $\lambda$
with marked optimal lambda}
\end{center}
\end{figure}

The impact of lambda on GCV can be visualized:
\begin{figure}[H]
\begin{center}
\includegraphics{figures/fig-GCV_LAMBDA}
\caption{GCV with respect to lmbda}
\end{center}
\end{figure}
Coefficients of selected model:
\begin{Schunk}
\begin{Soutput}
                           AG.LND.AGRI.K2    AG.LND.ARBL.HA.PC       AG.LND.ARBL.ZS 
        2.820289e-15         4.891585e-07        -7.116879e+00        -8.679093e-02 
      AG.LND.CROP.ZS       AG.LND.TOTL.K2       AG.PRD.CROP.XD       AG.PRD.FOOD.XD 
        8.121092e-03         4.340013e-08        -2.326247e-02         9.425381e-03 
      AG.PRD.LVSK.XD       AG.SRF.TOTL.K2       AG.YLD.CREL.KG       BM.GSR.INSF.ZS 
        6.884878e-02         4.132774e-08        -1.663944e-03        -9.167729e-02 
      BM.GSR.TRVL.ZS       BX.GSR.CMCP.ZS BX.KLT.DINV.WD.GD.ZS    EG.GDP.PUSE.KO.PP 
        5.776156e-02        -1.603808e-02         1.765687e-02        -3.708520e-01 
EG.GDP.PUSE.KO.PP.KD    EG.USE.COMM.KT.OE EG.USE.COMM.GD.PP.KD    EG.USE.ELEC.KH.PC 
       -4.498845e-01         1.204373e-06         1.260502e-03        -5.198139e-04 
   EN.ATM.CO2E.KD.GD       EN.ATM.CO2E.PC    EN.ATM.PM10.MC.M3       ER.H2O.INTR.K3 
        5.849059e-01        -7.249030e-01         5.265726e-02         6.155290e-05 
      ER.H2O.INTR.PC    FM.LBL.MQMY.GD.ZS    FS.AST.PRVT.GD.ZS       IC.CRD.PRVT.ZS 
        5.709176e-06        -8.496698e-03        -3.662162e-02        -9.630877e-02 
         IC.EXP.DURS       IC.LGL.CRED.XQ       NE.RSB.GNFS.ZS       NE.TRD.GNFS.ZS 
        1.238967e-01        -1.934809e-01        -1.876214e-02        -3.826836e-02 
\end{Soutput}
\end{Schunk}
RMSE computed using leave one out is
22.61. We see that biased estimator which
decreases the variance of estimator increasing the bias can give worse results
in terms of prediction compared to other methods.

\subsection{LASSO regression}
\begin{figure}[H]
\begin{center}
\includegraphics{figures/fig-FIG_LASSO}
\caption{Visualization of the coefficients paths for LASSO}
\end{center}
\end{figure}

Choosing the best subset of predictors in LASSO regresssion on the basis of
Mallows Cp criterion

\begin{figure}[H]
\begin{center}
\includegraphics{figures/fig-FIG_CP_LASSO}
\caption{Visualization of Cp for LASSO}
\end{center}
\end{figure}

So to select the best coefficient based on $C_p$ criterium we select the set
with the smallest $C_p$:
\begin{Schunk}
\begin{Sinput}
> (lasso.coef.best.idx<-as.numeric(which.min(model.lasso$Cp)))
\end{Sinput}
\begin{Soutput}
[1] 18
\end{Soutput}
\end{Schunk}

values of fitted coefficients (bi) in LASSO regression for
the chosen model

\begin{Schunk}
\begin{Sinput}
> model.lasso$beta[lasso.coef.best.idx,]	
\end{Sinput}
\begin{Soutput}
      AG.LND.AGRI.K2    AG.LND.ARBL.HA.PC       AG.LND.ARBL.ZS       AG.LND.CROP.ZS 
        1.632643e-07        -5.946752e+00        -1.059844e-01         0.000000e+00 
      AG.LND.TOTL.K2       AG.PRD.CROP.XD       AG.PRD.FOOD.XD       AG.PRD.LVSK.XD 
        0.000000e+00         0.000000e+00         0.000000e+00         3.405054e-02 
      AG.SRF.TOTL.K2       AG.YLD.CREL.KG       BM.GSR.INSF.ZS       BM.GSR.TRVL.ZS 
        0.000000e+00        -1.630761e-03         0.000000e+00         5.545624e-02 
      BX.GSR.CMCP.ZS BX.KLT.DINV.WD.GD.ZS    EG.GDP.PUSE.KO.PP EG.GDP.PUSE.KO.PP.KD 
        0.000000e+00         0.000000e+00         0.000000e+00        -9.882746e-01 
   EG.USE.COMM.KT.OE EG.USE.COMM.GD.PP.KD    EG.USE.ELEC.KH.PC    EN.ATM.CO2E.KD.GD 
        0.000000e+00         0.000000e+00        -6.132116e-04         3.826532e-01 
      EN.ATM.CO2E.PC    EN.ATM.PM10.MC.M3       ER.H2O.INTR.K3       ER.H2O.INTR.PC 
       -9.547260e-01         4.250566e-02         0.000000e+00         0.000000e+00 
   FM.LBL.MQMY.GD.ZS    FS.AST.PRVT.GD.ZS       IC.CRD.PRVT.ZS          IC.EXP.DURS 
        0.000000e+00        -2.093679e-02        -1.241605e-01         1.477372e-01 
      IC.LGL.CRED.XQ       NE.RSB.GNFS.ZS       NE.TRD.GNFS.ZS 
        0.000000e+00         0.000000e+00        -5.170877e-02 
\end{Soutput}
\end{Schunk}

Choosing best predictors in LASSO regresssion on the basis of
crossvalidation:

\begin{figure}[H]
\begin{center}
\includegraphics{figures/fig-FIG_MSE_LASSO}
\caption{Visualisation of CV MSE}
\end{center}
\end{figure}
Getting coefficient for the minimum MSE:
\begin{Schunk}
\begin{Sinput}
> frac<-kaggle.cv.lars$index[kaggle.cv.lars$cv==min(kaggle.cv.lars$cv)]	
> predict.lars(model.lasso, type="coefficients", mode="fraction", s=frac)$coef
\end{Sinput}
\begin{Soutput}
      AG.LND.AGRI.K2    AG.LND.ARBL.HA.PC       AG.LND.ARBL.ZS       AG.LND.CROP.ZS 
        1.075583e-06        -1.172119e+01        -1.598687e-01        -2.218017e-02 
      AG.LND.TOTL.K2       AG.PRD.CROP.XD       AG.PRD.FOOD.XD       AG.PRD.LVSK.XD 
        0.000000e+00        -6.754011e-02         0.000000e+00         1.710303e-02 
      AG.SRF.TOTL.K2       AG.YLD.CREL.KG       BM.GSR.INSF.ZS       BM.GSR.TRVL.ZS 
        1.200951e-07        -2.499797e-03        -1.686797e-01         1.981832e-01 
      BX.GSR.CMCP.ZS BX.KLT.DINV.WD.GD.ZS    EG.GDP.PUSE.KO.PP EG.GDP.PUSE.KO.PP.KD 
       -5.779094e-03         9.272045e-02        -3.228319e-01        -7.886943e-01 
   EG.USE.COMM.KT.OE EG.USE.COMM.GD.PP.KD    EG.USE.ELEC.KH.PC    EN.ATM.CO2E.KD.GD 
        3.794306e-07         0.000000e+00        -1.021712e-03         1.092142e+00 
      EN.ATM.CO2E.PC    EN.ATM.PM10.MC.M3       ER.H2O.INTR.K3       ER.H2O.INTR.PC 
       -9.807543e-01         6.430192e-02         0.000000e+00         2.456967e-05 
   FM.LBL.MQMY.GD.ZS    FS.AST.PRVT.GD.ZS       IC.CRD.PRVT.ZS          IC.EXP.DURS 
       -2.968208e-02         0.000000e+00        -1.165076e-01         8.675806e-02 
      IC.LGL.CRED.XQ       NE.RSB.GNFS.ZS       NE.TRD.GNFS.ZS 
        0.000000e+00         1.724057e-02        -8.841993e-02 
\end{Soutput}
\end{Schunk}

\subsection{Robust regression: M-estimators and Least Trimmed Squares}
Methods that deals with vialoted assumptions of the linear regression like
outliers, heterscedacticity of variance, fatter tails of errors. These
methods are usefull when automatic and quick model fitting is required or
to compare to LS model for validation (If they differ the source of
dissimilarity should be investigated). Here M-estimators apply the method with
the Huber function.
The M-estimators uses special function on residuals when minimizing sum of those (possibly different
than quadratic like in case of the LS). The Least Trimmed Squares ignores the
biggest residuals in the optimization process.  Both models are fitted using
predictors selected by VIF method so we are not impacted by colinearity.  
Comparing those coeffcients to the LS model allows us to check what is the
influence of the remaining outliers:
\begin{Schunk}
\begin{Soutput}
                               VIF         M-est           LTS
(Intercept)           1.183955e+02  1.114423e+02  1.120662e+02
AG.LND.AGRI.K2        1.598293e-06  7.333228e-07  3.650273e-05
AG.LND.ARBL.HA.PC    -1.253239e+01 -8.574811e+00  8.288523e+00
AG.LND.ARBL.ZS       -1.745265e-01 -1.611377e-01 -1.186622e-01
AG.LND.CROP.ZS       -1.570192e-01 -1.044620e-01 -2.215589e-01
AG.PRD.CROP.XD       -1.324605e-01 -1.448184e-01 -3.320323e-02
AG.PRD.LVSK.XD        3.700824e-02  7.421651e-02  3.672683e-02
AG.YLD.CREL.KG       -2.988844e-03 -2.492744e-03  1.294025e-03
BM.GSR.INSF.ZS       -3.661819e-01 -2.222917e-01 -2.817695e-01
BM.GSR.TRVL.ZS        2.215913e-01  2.072900e-01 -2.731435e-01
BX.GSR.CMCP.ZS       -4.889663e-02 -3.623866e-02  3.490963e-02
BX.KLT.DINV.WD.GD.ZS  1.123641e-01  8.797489e-02 -4.467767e-01
EG.GDP.PUSE.KO.PP    -1.061146e+00 -1.015587e+00 -7.831947e-01
EG.USE.COMM.KT.OE     2.605741e-06  4.210267e-06 -6.894729e-05
EG.USE.ELEC.KH.PC    -1.009742e-03 -7.513840e-04  2.430462e-04
EN.ATM.CO2E.KD.GD     1.704628e+00  1.474666e+00  4.844412e+00
EN.ATM.CO2E.PC       -1.267936e+00 -1.507218e+00 -4.821782e+00
EN.ATM.PM10.MC.M3     7.966807e-02  6.889363e-02 -9.204514e-02
ER.H2O.INTR.K3       -1.611311e-04  1.176023e-04 -5.023729e-05
ER.H2O.INTR.PC        3.734656e-05  2.162018e-05 -1.248380e-04
FS.AST.PRVT.GD.ZS    -1.720066e-02 -2.576060e-02  1.282050e-01
IC.CRD.PRVT.ZS       -1.073238e-01 -1.140820e-01 -1.521718e-01
IC.EXP.DURS           4.547959e-02  5.496453e-02  4.796941e-02
IC.LGL.CRED.XQ       -1.526177e-02 -1.773783e-01 -4.785446e-01
NE.RSB.GNFS.ZS        3.659690e-02  5.853014e-02  9.089325e-02
NE.TRD.GNFS.ZS       -1.012795e-01 -7.978937e-02 -6.599761e-02
\end{Soutput}
\end{Schunk}
As we can see in most of the cases the robust regression methods does not change
coefficients considerably so we can conclude that outliers does not change model
too much.

\subsection{PCA}
PCA tries to rotate model matrix into ortogonality hence simplifying testing and
interpretation. I use prcomp function which should be more acurate than
princomp (uses SVD).
After performing PCA on kaggle data we obtain standard
deviations for principal components:
\begin{figure}[H]
\begin{center}
\includegraphics{figures/fig-FIG_PCA}
\caption{Principal components' variance of kaggle data}
\end{center}
\end{figure}
We see that first few components contain the majority of variation in the data.

Visualizing 3 first componets as composition of orginal predictors:
\begin{figure}[H]
\begin{center}
\includegraphics{figures/fig-FIG_PCA_COMP}
\caption{3 first principal components as composition of original predicotrs.}
\end{center}
\end{figure}
It can be seen that first 3 principal components somehow weigh orginal
predictors in differeent ways but it is difficult to find the systematic pattern
there. 

PCA can also be used to discover extreme observations in the data. Getting the
biggest value for the first component we can find entreme observation: 
\begin{Schunk}
\begin{Sinput}
> #Extream observation
> extreme.idx<-which.max(kaggle.pc$x[,1])
> kaggle.data$country[extreme.idx]
\end{Sinput}
\begin{Soutput}
[1] Iraq
87 Levels: Afghanistan Albania Antigua & Barbuda Argentina Armenia Austria Azerbaijan ... Venezuela
\end{Soutput}
\begin{Sinput}
> kaggle.data.scaled<-scale(kaggle.data[variable.names])
> kaggle.data.scaled[extreme.idx,]
\end{Sinput}
\begin{Soutput}
      AG.LND.AGRI.K2    AG.LND.ARBL.HA.PC       AG.LND.ARBL.ZS       AG.LND.CROP.ZS 
         -0.30791251          -0.11299188          -0.18701738          -0.53570288 
      AG.LND.TOTL.K2       AG.PRD.CROP.XD       AG.PRD.FOOD.XD       AG.PRD.LVSK.XD 
         -0.24024095           0.26282849          -0.04531691           1.21019539 
      AG.SRF.TOTL.K2       AG.YLD.CREL.KG       BM.GSR.INSF.ZS       BM.GSR.TRVL.ZS 
         -0.24513837          -1.07499847           1.44074075          -1.11903857 
      BX.GSR.CMCP.ZS BX.KLT.DINV.WD.GD.ZS    EG.GDP.PUSE.KO.PP EG.GDP.PUSE.KO.PP.KD 
          1.90884956          -0.25926840          -1.08844310          -0.92712124 
   EG.USE.COMM.KT.OE EG.USE.COMM.GD.PP.KD    EG.USE.ELEC.KH.PC    EN.ATM.CO2E.KD.GD 
         -0.22867533           1.65595716          -0.42149955           2.35924312 
      EN.ATM.CO2E.PC    EN.ATM.PM10.MC.M3       ER.H2O.INTR.K3       ER.H2O.INTR.PC 
         -0.29028135           3.74026447          -0.36467701          -0.34207405 
   FM.LBL.MQMY.GD.ZS    FS.AST.PRVT.GD.ZS       IC.CRD.PRVT.ZS          IC.EXP.DURS 
         -0.26484286          -0.97149414          -0.72584404           4.33963016 
      IC.LGL.CRED.XQ       NE.RSB.GNFS.ZS       NE.TRD.GNFS.ZS     Corruption.Index 
         -0.97051999           0.44970581          -1.03782687           1.61652583 
\end{Soutput}
\end{Schunk}

\subsection{PCR}
Principal component regression builds linear model based on the principal
components computed like in the previous section. 
Performance of this model can be checked by computing RMSE (I am using here
leave one out crossvalidation) for models built from various number of most
important components: 
\begin{figure}[H]
\begin{center}
\includegraphics{figures/fig-FIG_PCR}
\caption{RMSEP $\sim$ number of componets using leave one out crossvalidation}
\end{center}
\end{figure}
We see that the best performance is achieved for 15 PCs with RMSE: 
\begin{Schunk}
\begin{Soutput}
[1] 17.4021
\end{Soutput}
\end{Schunk}

Visualising the components as the linear function of orginal predictors.
\begin{figure}[H]
\begin{center}
\includegraphics{figures/fig-FIG_PCR_COMPONENTS}
\caption{15 first PCR components as function of orginal
predictors}
\end{center}
\end{figure}
We can see that all components select similar orginal predictors but weigh them
with different intensity.

Checking if model doesn't violate OLS assumptions:
\begin{figure}[H]
\begin{center}
\includegraphics{figures/fig-DIAGN_PCR}
\caption{Diagnostic plots for PCR model with 15 components}
\end{center}
\end{figure}
We see that model is conformant with OLS assumptions like normal distribution of
residuals, homogenous variance.

\subsection{PLSR}
Partial least squares regression is similar to PCR (builds predictors based on
linear combination of original predictors) with one important difference that
it uses information how predictors influence response variable (where PCR
ignores this information).
Performance of this model can be checked by computing RMSE (I am using here
leave one out crossvalidation) for models built from various number of most
important components: 
\begin{figure}[H]
\begin{center}
\includegraphics{figures/fig-FIG_PLSR}
\caption{RMSEP $\sim$ number of componets}
\end{center}
\end{figure}

We see that the best performance is achieved for 11 components with RMSE: 
\begin{Schunk}
\begin{Soutput}
[1] 17.36973
\end{Soutput}
\end{Schunk}
So using PLSR allowed us to decrease number of predictors and error as well
compared to PCR.

Visualising the components as the linear function of orginal predictors.
\begin{figure}[H]
\begin{center}
\includegraphics{figures/fig-FIG_PLSR_PCA}
\caption{11 first PLSR components as function of orginal predictors}
\end{center}
\end{figure}
We can see that all components select similar orginal predictors but weigh them
with different intensity.

Checking if model doesn't violate OLS assumptions:
\begin{figure}[H]
\begin{center}
\includegraphics{figures/fig-DIAGN_PLSR}
\caption{Diagnostic plots for PLSR model with 11 components}
\end{center}
\end{figure}
We see that model is conformant with OLS assumptions like normal distribution of
residuals, homogenous variance.

\section{Nonlinear regression}

\subsection{Regression tree}

Tree having the minimum xerror (crossvalidated error: ratio of $R^{CV}(T)$ and
SSE for root) has 3 splits
and coresponding SE=0.115:
\begin{Schunk}
\begin{Soutput}
Regression tree:
rpart(formula = Corruption.Index ~ ., data = kaggle.data[variable.names])

Variables actually used in tree construction:
[1] AG.LND.ARBL.ZS       AG.LND.CROP.ZS       BX.KLT.DINV.WD.GD.ZS EG.GDP.PUSE.KO.PP.KD
[5] EG.USE.ELEC.KH.PC    EN.ATM.CO2E.PC       IC.EXP.DURS         

Root node error: 40413/83 = 486.9

n= 83 

        CP nsplit rel error  xerror    xstd
1 0.419584      0   1.00000 1.01450 0.14153
2 0.180685      1   0.58042 0.82332 0.14397
3 0.061492      2   0.39973 0.73449 0.14563
4 0.034589      3   0.33824 0.67533 0.11516
5 0.033165      4   0.30365 0.70010 0.11133
6 0.028088      5   0.27048 0.68418 0.11126
7 0.025676      6   0.24240 0.68418 0.11126
8 0.010000      7   0.21672 0.67899 0.11425
\end{Soutput}
\end{Schunk}
The same selection process depicted graphically:
\begin{figure}[H]
\begin{center}
\includegraphics{figures/fig-RPART_TREE}
\caption{Crossvalidated error as a function of number of splits}
\end{center}
\end{figure}

Built tree:
\begin{figure}[H]
\begin{center}
\includegraphics{figures/fig-FIG_RPART_TREE}
\caption{Regression tree}
\end{center}
\end{figure}

\subsection{Moving average estimator of regression function}
 Fitting models with different values of span, because only 4 predictors are
 allowed:

It can be seen that small values of smoothing parameter results in the model
that tries to fits the data too closely and is over-fitted. 
\begin{figure}[H]
\begin{center}
\includegraphics{figures/fig-FIG_MOVING_AVG}
\caption{moving average model for different values of smoothing parameter}
\end{center}
\end{figure}

\subsection{Local linear estimator of regression function}

Fitting models with different values of span:

It can be seen that small values of smoothing parameter results in the model
that tries to fits the data too closely and is over-fitted. 
\begin{figure}[H]
\begin{center}
\includegraphics{figures/fig-FIG_LOC_LIN_EST}
\caption{local quadratic estimator model for different values of smoothing parameter}
\end{center}
\end{figure}

\subsection{Additive model}

Selected additive model based on forward procedure with t-test: 
\begin{Schunk}
\begin{Sinput}
> summary(model.additive)
\end{Sinput}
\begin{Soutput}
Family: gaussian 
Link function: identity 

Formula:
Corruption.Index ~ s(EG.USE.ELEC.KH.PC) + s(IC.EXP.DURS) + s(EG.GDP.PUSE.KO.PP.KD) + 
    s(AG.LND.ARBL.ZS) + s(IC.CRD.PRVT.ZS) + s(NE.TRD.GNFS.ZS)
<environment: 0x7fe2747d05f8>

Parametric coefficients:
            Estimate Std. Error t value Pr(>|t|)    
(Intercept)   68.913      1.307   52.73   <2e-16 ***
---
Signif. codes:  0 '***' 0.001 '**' 0.01 '*' 0.05 '.' 0.1 ' ' 1

Approximate significance of smooth terms:
                          edf Ref.df      F  p-value    
s(EG.USE.ELEC.KH.PC)    2.240  2.705 20.352 3.17e-09 ***
s(IC.EXP.DURS)          1.000  1.000  3.795  0.05534 .  
s(EG.GDP.PUSE.KO.PP.KD) 5.723  6.712  3.075  0.00771 ** 
s(AG.LND.ARBL.ZS)       1.000  1.000  6.891  0.01058 *  
s(IC.CRD.PRVT.ZS)       1.000  1.000  9.029  0.00366 ** 
s(NE.TRD.GNFS.ZS)       1.000  1.000  7.421  0.00809 ** 
---
Signif. codes:  0 '***' 0.001 '**' 0.01 '*' 0.05 '.' 0.1 ' ' 1

R-sq.(adj) =  0.712   Deviance explained = 75.4%
GCV score = 167.99  Scale est. = 141.75    n = 83
\end{Soutput}
\end{Schunk}

\begin{figure}[H]
\begin{center}
\includegraphics{figures/fig-FIG_SELECTED_ADDITIVE_MODEL}
\caption{Plot of fuctions that constitute additive model}
\end{center}
\end{figure}

\section{Prediction}
The ultimate test for the model is the prediction of the out of sample data. I
use United Kindom data from 2007 which is not included in the data set but its
data can be found on Internet \cite{worldbankData}.Of course the model has been
built from the data gathered in unknown year so we have to assume that the model
is still valid in 2007 (Qualitative extrapolation \cite{Faraway}). Also checking
model by predicting outcome variable that can be verified with real life data
gives better feel of model quality than for example trying to explain the value
of cooeficients \cite{Faraway}. 
The prediction quality also depends on the
Mahalanobis distance of the new observation from the cases used to build the
model which equals: 6.28. Because
the mean value of observation distances from their center equals
5.35 and the standard
deviation of those distances equals
1.43 we can predict the index for UK.
The real value of Corruption.Index for UK is
34.1 \cite{failedIndex2007}.

\begin{longtable}{ l | p{10cm} }
\textbf{Model} & \textbf{Prediction} \\
\hline 
Full model &
\begin{Schunk}
\begin{Soutput}
      fit      lwr      upr
 49.35854 18.27891 80.43818
\end{Soutput}
\end{Schunk}
\\
VIF model &
\begin{Schunk}
\begin{Soutput}
      fit      lwr      upr
 44.32452 14.23575 74.41329
\end{Soutput}
\end{Schunk}
\\
Backward t-test &
\begin{Schunk}
\begin{Soutput}
      fit      lwr      upr
 43.74599 16.92364 70.56835
\end{Soutput}
\end{Schunk}
\\
Backward t-test with quadratic term &
\begin{Schunk}
\begin{Soutput}
      fit      lwr      upr
 45.77669 19.88974 71.66364
\end{Soutput}
\end{Schunk}
\\
Forward t-test &
\begin{Schunk}
\begin{Soutput}
      fit      lwr      upr
 47.08985 20.28054 73.89915
\end{Soutput}
\end{Schunk}
\\
Backward AIC &
\begin{Schunk}
\begin{Soutput}
      fit      lwr      upr
 47.92342 20.89829 74.94854
\end{Soutput}
\end{Schunk}
\\
Forward AIC &
\begin{Schunk}
\begin{Soutput}
      fit      lwr      upr
 41.28042 14.60942 67.95141
\end{Soutput}
\end{Schunk}
\\
Forward AIC with quadratic term&
\begin{Schunk}
\begin{Soutput}
      fit      lwr      upr
 41.93812 16.16607 67.71017
\end{Soutput}
\end{Schunk}
\\
Backward BIC &
\begin{Schunk}
\begin{Soutput}
     fit      lwr      upr
 39.8606 11.67301 68.04819
\end{Soutput}
\end{Schunk}
\\
Forward BIC &
\begin{Schunk}
\begin{Soutput}
      fit      lwr      upr
 48.88516 21.56853 76.20179
\end{Soutput}
\end{Schunk}
\\
Forward BIC with quadratic term&
\begin{Schunk}
\begin{Soutput}
      fit      lwr      upr
 45.87882 18.91644 72.84119
\end{Soutput}
\end{Schunk}
\\
Adjusted R2&
\begin{Schunk}
\begin{Soutput}
      fit      lwr      upr
 49.04343 21.75061 76.33624
\end{Soutput}
\end{Schunk}
\\
Adjusted R2 with quadratic term&
\begin{Schunk}
\begin{Soutput}
      fit      lwr      upr
 49.81643 23.02251 76.61035
\end{Soutput}
\end{Schunk}
\\
Cp&
\begin{Schunk}
\begin{Soutput}
      fit      lwr     upr
 40.74901 13.93553 67.5625
\end{Soutput}
\end{Schunk}
\\
Cp with quadratic term&
\begin{Schunk}
\begin{Soutput}
      fit      lwr      upr
 42.23055 16.13695 68.32415
\end{Soutput}
\end{Schunk}
\\
PCR&
\begin{Schunk}
\begin{Soutput}
38.87232
\end{Soutput}
\end{Schunk}
\\
PLSR&
\begin{Schunk}
\begin{Soutput}
39.45651
\end{Soutput}
\end{Schunk}
\\
Lasso &
43.34699\\
Ridge &
41.98209\\
Least trimmed squares &
45.88068\\
M-estimator(Huber) &
\begin{Schunk}
\begin{Soutput}
      fit      lwr     upr
 42.47002 20.80984 64.1302
\end{Soutput}
\end{Schunk}
\\
Regression tree &
\begin{Schunk}
\begin{Soutput}
49.6625
\end{Soutput}
\end{Schunk}

\\
Additive &
48.59347\\
\caption{Prdicted index for United Kindom for different models}
\end{longtable}

We can see that predcting new values gives very wide intervals which contain the
true value.

\section{Crossvalidation and final model selection}\label{sec:cross} 
I decided to select model based on leave one out crossvalidation to choose
model with the best potential to predict index for new data. 
Each model is recomputed with one observation left out (for each observation)
and then RMSE (root mean square error is computed) of all models is calculated.
I select model with the smallest crossvalidation RMSE.



\begin{longtable}{ l | p{10cm} }
\textbf{Model} & \textbf{Prediction} \\
\hline 
Full model &
\begin{Schunk}
\begin{Soutput}
17.43758
\end{Soutput}
\end{Schunk}
\\
VIF model &
\begin{Schunk}
\begin{Soutput}
18.06011
\end{Soutput}
\end{Schunk}
\\
Backward t-test &
\begin{Schunk}
\begin{Soutput}
14.3501
\end{Soutput}
\end{Schunk}
\\
Backward t-test with quadratic term &
\begin{Schunk}
\begin{Soutput}
16.57995
\end{Soutput}
\end{Schunk}
\\
Forward t-test &
\begin{Schunk}
\begin{Soutput}
14.18182
\end{Soutput}
\end{Schunk}
\\
Backward AIC &
\begin{Schunk}
\begin{Soutput}
13.93362
\end{Soutput}
\end{Schunk}
\\
Forward AIC &
\begin{Schunk}
\begin{Soutput}
14.37905
\end{Soutput}
\end{Schunk}
\\
Forward AIC with quadratic term&
\begin{Schunk}
\begin{Soutput}
13.69938
\end{Soutput}
\end{Schunk}
\\
Backward BIC &
\begin{Schunk}
\begin{Soutput}
14.17277
\end{Soutput}
\end{Schunk}
\\
Forward BIC &
\begin{Schunk}
\begin{Soutput}
14.32132
\end{Soutput}
\end{Schunk}
\\
Forward BIC with quadratic term&
\begin{Schunk}
\begin{Soutput}
14.17384
\end{Soutput}
\end{Schunk}
\\
Adjusted R2&
\begin{Schunk}
\begin{Soutput}
14.45955
\end{Soutput}
\end{Schunk}
\\
Adjusted R2 with quadratic term&
\begin{Schunk}
\begin{Soutput}
14.10968
\end{Soutput}
\end{Schunk}
\\
Cp&
\begin{Schunk}
\begin{Soutput}
14.0058
\end{Soutput}
\end{Schunk}
\\
Cp with quadratic term&
\begin{Schunk}
\begin{Soutput}
13.54951
\end{Soutput}
\end{Schunk}
\\
Principal component regression&
\begin{Schunk}
\begin{Soutput}
17.4021
\end{Soutput}
\end{Schunk}
\\
Partial least squares regression&
\begin{Schunk}
\begin{Soutput}
17.36973
\end{Soutput}
\end{Schunk}
\\
Ridge regression&
\begin{Schunk}
\begin{Soutput}
22.605
\end{Soutput}
\end{Schunk}
\\
Regression tree&
\begin{Schunk}
\begin{Soutput}
18.38257
\end{Soutput}
\end{Schunk}
\\
\caption{Leave one out crossvalidation RMSE for all models}
\label{tab:cross_valid}
\end{longtable}

Besed on the Table-\ref{tab:cross_valid} I select model 
obtainen by the Cp with quadratic term because of the smallest corss
validation error. This model selection (based on Cp criterion) aims at
choosing the best model for prediction tasks
(selection procedure in this case minimizes average mean sqare error). The Cp
model with quadratic term:

\begin{Schunk}
\begin{Soutput}
                          Estimate  Std. Error t value  Pr(>|t|)    
(Intercept)             1.5606e+02  1.6991e+01  9.1847 3.589e-13 ***
AG.LND.ARBL.HA.PC      -1.6057e+01  5.5692e+00 -2.8831  0.005406 ** 
AG.LND.ARBL.ZS         -2.8621e-01  1.1041e-01 -2.5923  0.011875 *  
AG.LND.CROP.ZS         -2.8151e-01  2.0406e-01 -1.3796  0.172679    
AG.PRD.CROP.XD         -2.4866e-01  1.0277e-01 -2.4197  0.018480 *  
AG.SRF.TOTL.K2          6.0943e-07  6.4591e-07  0.9435  0.349079    
AG.YLD.CREL.KG         -2.7760e-03  1.0115e-03 -2.7446  0.007915 ** 
BM.GSR.INSF.ZS         -2.0882e-01  2.0494e-01 -1.0189  0.312196    
BM.GSR.TRVL.ZS          4.0353e-01  1.3924e-01  2.8982  0.005182 ** 
BX.KLT.DINV.WD.GD.ZS    3.3210e-01  1.3861e-01  2.3959  0.019612 *  
EG.GDP.PUSE.KO.PP      -5.6541e+00  2.1399e+00 -2.6422  0.010413 *  
EG.USE.ELEC.KH.PC      -1.2730e-03  3.7490e-04 -3.3956  0.001200 ** 
EN.ATM.CO2E.KD.GD       1.0075e+00  1.0128e+00  0.9947  0.323753    
EN.ATM.CO2E.PC         -1.2048e+00  4.3729e-01 -2.7553  0.007689 ** 
EN.ATM.PM10.MC.M3       7.3750e-02  4.5031e-02  1.6378  0.106535    
FM.LBL.MQMY.GD.ZS      -1.1505e-01  6.3879e-02 -1.8010  0.076561 .  
FS.AST.PRVT.GD.ZS       7.1091e-02  6.3217e-02  1.1246  0.265112    
IC.CRD.PRVT.ZS         -1.0506e-01  5.6317e-02 -1.8655  0.066848 .  
NE.TRD.GNFS.ZS         -1.1873e-01  4.0661e-02 -2.9199  0.004876 ** 
I(EG.GDP.PUSE.KO.PP^2)  2.2378e-01  1.0514e-01  2.1285  0.037278 *  
---
Signif. codes:  0 '***' 0.001 '**' 0.01 '*' 0.05 '.' 0.1 ' ' 1
\end{Soutput}
\end{Schunk}

\section{Summary}
Models also fulfill explanatory function for the data.

\vspace{8cm}

\begingroup
\begin{table}[H]
\fontsize{7pt}{8pt}\selectfont
\begin{center}
  \advance\leftskip-0.5cm
  \begin{tabular}{ l |
p{0.1cm} p{0.1cm} p{0.1cm} p{0.1cm} p{0.1cm} p{0.1cm} p{0.1cm} p{0.1cm} p{0.1cm} p{0.1cm} p{0.1cm} p{0.1cm} p{0.1cm} p{0.1cm} p{0.1cm} p{0.1cm} p{0.1cm} p{0.1cm} p{0.1cm} p{0.1cm} p{0.1cm} p{0.1cm} p{0.1cm} p{0.1cm} p{0.1cm} p{0.1cm} p{0.1cm} p{0.1cm} p{0.1cm} p{0.1cm} p{0.1cm}  p{0.5cm} p{0.5cm}}
    \textbf{Method} &
\begin{rotate}{90} \textbf{AG.LND.AGRI.K2} \end{rotate}&
\begin{rotate}{90} \textbf{AG.LND.ARBL.HA.PC} \end{rotate}&
\begin{rotate}{90} \textbf{AG.LND.ARBL.ZS} \end{rotate}&
\begin{rotate}{90} \textbf{AG.LND.CROP.ZS} \end{rotate}&
\begin{rotate}{90} \textbf{AG.LND.TOTL.K2} \end{rotate}&
\begin{rotate}{90} \textbf{AG.PRD.CROP.XD} \end{rotate}&
\begin{rotate}{90} \textbf{AG.PRD.FOOD.XD} \end{rotate}&
\begin{rotate}{90} \textbf{AG.PRD.LVSK.XD} \end{rotate}&
\begin{rotate}{90} \textbf{AG.SRF.TOTL.K2} \end{rotate}&
\begin{rotate}{90} \textbf{AG.YLD.CREL.KG} \end{rotate}&
\begin{rotate}{90} \textbf{BM.GSR.INSF.ZS} \end{rotate}&
\begin{rotate}{90} \textbf{BM.GSR.TRVL.ZS} \end{rotate}&
\begin{rotate}{90} \textbf{BX.GSR.CMCP.ZS} \end{rotate}&
\begin{rotate}{90} \textbf{BX.KLT.DINV.WD.GD.ZS} \end{rotate}&
\begin{rotate}{90} \textbf{EG.GDP.PUSE.KO.PP} \end{rotate}&
\begin{rotate}{90} \textbf{EG.GDP.PUSE.KO.PP.KD} \end{rotate}&
\begin{rotate}{90} \textbf{EG.USE.COMM.KT.OE} \end{rotate}&
\begin{rotate}{90} \textbf{EG.USE.COMM.GD.PP.KD} \end{rotate}&
\begin{rotate}{90} \textbf{EG.USE.ELEC.KH.PC} \end{rotate}&
\begin{rotate}{90} \textbf{EN.ATM.CO2E.KD.GD} \end{rotate}&
\begin{rotate}{90} \textbf{EN.ATM.CO2E.PC} \end{rotate}&
\begin{rotate}{90} \textbf{EN.ATM.PM10.MC.M3} \end{rotate}&
\begin{rotate}{90} \textbf{ER.H2O.INTR.K3} \end{rotate}&
\begin{rotate}{90} \textbf{ER.H2O.INTR.PC} \end{rotate}&
\begin{rotate}{90} \textbf{FM.LBL.MQMY.GD.ZS} \end{rotate}&
\begin{rotate}{90} \textbf{FS.AST.PRVT.GD.ZS} \end{rotate}&
\begin{rotate}{90} \textbf{IC.CRD.PRVT.ZS} \end{rotate}&
\begin{rotate}{90} \textbf{IC.EXP.DURS} \end{rotate}&
\begin{rotate}{90} \textbf{IC.LGL.CRED.XQ} \end{rotate}&
\begin{rotate}{90} \textbf{NE.RSB.GNFS.ZS} \end{rotate}&
\begin{rotate}{90} \textbf{NE.TRD.GNFS.ZS} \end{rotate}&    \begin{rotate}{90} \textbf{Systematic part exmplained}\end{rotate}&
    \begin{rotate}{90} \textbf{Sigma}\end{rotate}\\
    \hline 
    
Full model&+&\textcolor{Red}{-}&-&-&-&-&+&-&+&\textcolor{Red}{-}&-&+&-&\textcolor{Red}{+}&-&-&-&-&\textcolor{Red}{-}&+&-&+&-&+&-&+&-&+&-&+&\textcolor{Red}{-}&0.639&13.333 \\    
VIF Backward&+&-&-&-&&-&&-&&\textcolor{Red}{-}&-&+&-&\textcolor{Red}{+}&\textcolor{Red}{-}&&-&&-&+&-&+&-&+&&+&-&+&-&+&\textcolor{Red}{-}&0.638&13.361 \\    
Backward t-test&&\textcolor{Red}{-}&\textcolor{Red}{-}&&&-&&&&\textcolor{Red}{-}&&\textcolor{Red}{+}&&+&\textcolor{Red}{-}&&&&\textcolor{Red}{-}&+&\textcolor{Red}{-}&+&&&-&&\textcolor{Red}{-}&&&&\textcolor{Red}{-}&0.670&12.539 \\    
Forward t-test&&&\textcolor{Red}{-}&&&&&&&&&&&&&\textcolor{Red}{-}&&&\textcolor{Red}{-}&&-&&&&&&\textcolor{Red}{-}&\textcolor{Red}{+}&&&\textcolor{Red}{-}&0.646&13.041 \\
Backward AIC&&\textcolor{Red}{-}&\textcolor{Red}{-}&&&-&&&&\textcolor{Red}{-}&-&\textcolor{Red}{+}&&\textcolor{Red}{+}&\textcolor{Red}{-}&&&&\textcolor{Red}{-}&+&\textcolor{Red}{-}&\textcolor{Red}{+}&&&\textcolor{Red}{-}&+&\textcolor{Red}{-}&&&&\textcolor{Red}{-}&0.689&12.389 \\
Forward AIC&&\textcolor{Red}{-}&-&&&&&&&\textcolor{Red}{-}&&\textcolor{Red}{+}&&\textcolor{Red}{+}&&\textcolor{Red}{-}&&&\textcolor{Red}{-}&&-&+&&+&-&&\textcolor{Red}{-}&+&&&\textcolor{Red}{-}&0.680&12.554 \\
Backward BIC&&&&&&&&&&\textcolor{Red}{-}&&&&&\textcolor{Red}{-}&&&&\textcolor{Red}{-}&&-&\textcolor{Red}{+}&&&&&\textcolor{Red}{-}&&&&&0.619&13.519 \\
Forward BIC&&&&&&&&&&&&&&&&\textcolor{Red}{-}&&&\textcolor{Red}{-}&&-&&&&&&\textcolor{Red}{-}&\textcolor{Red}{+}&&&\textcolor{Red}{-}&0.631&13.319 \\
Adj R2&&\textcolor{Red}{-}&\textcolor{Red}{-}&-&&\textcolor{Red}{-}&&&+&\textcolor{Red}{-}&-&\textcolor{Red}{+}&&\textcolor{Red}{+}&\textcolor{Red}{-}&&&&\textcolor{Red}{-}&+&\textcolor{Red}{-}&\textcolor{Red}{+}&&&\textcolor{Red}{-}&+&\textcolor{Red}{-}&&&&\textcolor{Red}{-}&0.691&12.349 \\
Cp&&\textcolor{Red}{-}&&&&&&&&\textcolor{Red}{-}&&+&&\textcolor{Red}{+}&&\textcolor{Red}{-}&&&\textcolor{Red}{-}&&&\textcolor{Red}{+}&&\textcolor{Red}{+}&\textcolor{Red}{-}&&\textcolor{Red}{-}&&&&\textcolor{Red}{-}&0.674&12.675 \\
    
 \end{tabular}
 \caption{Models and selected predictors. "+" means that cooeficient in the
 model is positive and "-" means the cooeficient is negative. Lack of sign means
 that predictor is not used in the model. Red color means that coefficient is
 significant.}
 \end{center}
 \end{table}
 
We can see from the table how particular predictors influence the
Corruption.Index (holding all other predictors constant). Of course there is a
danger of lurking variables and we cannot make strong conclusions but we can at
least try to discover some trends:
\begin{itemize}
  \item AG.LND.AGRI.K2 positively influences Corruption.Index so we can assume
  that more agricultural land then country could be more corrupted.
  \item AG.LND.ARBL.HA.PC negativel influences Corruption.Index so we can assume
  that more arable land per person then country could be less corrupted
  \item AG.YLD.CREL.KG negatively influences Corruption.Index so we can assume
  that more Cereal yield then country could be less corrupted (better
  efficiency)
  \item BM.GSR.TRVL.ZS positively influences Corruption.Index so we can assume
  that if country has more turism as share of its economy then country could be
  more corrupted. (southern european countries)
  \item BX.KLT.DINV.WD.GD.ZS positively influences Corruption.Index so we can assume
  that bigger foreign direct investment as share of its economy then country
  could be more corrupted. (looks like coruption helps direct investment?)
  \item EG.GDP.PUSE.KO.PP negatively influences Corruption.Index so we can
  assume that more GDP per unit of energy use then country could be less corrupted 
  (more corrupted countries have less energy hungry economies).
  \item EG.USE.ELEC.KH.PC negatively influences Corruption.Index so we can
  assume that more electric power consumption per capita then country could be less
  corrupted (more corrupted countries have less energy hungry economies)
  \item EN.ATM.CO2E.PC CO2 negatively influences Corruption.Index so we can
  assume that bigger emissions per capita the country could be less corrupted 
  (more corrupted countries have less energy hungry economies)
  \item EN.ATM.PM10.MC.M3 positively influences Corruption.Index so we can assume
  the bigger polution the country could be more corrupted.
  \item IC.CRD.PRVT.ZS negatively influences Corruption.Index so we can
  assume the more state information on business the country could be less corrupted 
  \item IC.EXP.DURS positively influences Corruption.Index so we can assume the
  more time to export the country could be more corrupted 
  \item NE.TRD.GNFS.ZS negatively influences Corruption.Index so we can assume
  the more trade as proportion of GDP the country could be less corrupted
\end{itemize}
We can see that all those assosiations make logical sense.

\section{Methods that failed}
Here I note the methods that I tried but haven't imroved the model
\begin{itemize}
  \item Taking log of predictor which is right skewed
\end{itemize}
 
\begin{thebibliography}{9}
  \bibitem{failedIndex2007}
  \url{http://ffp.statesindex.org/rankings-2007-sortable} Corruption index 2007
  \bibitem{worldbankData}
  \url{http://data.worldbank.org/} Worldbank data
  \bibitem{Faraway} Julian J. Faraway "Linear Models with R"
  \bibitem{WikiFailedStatesIndex}
  \url{http://en.wikipedia.org/wiki/List_of_countries_by_Failed_States_Index}
\end{thebibliography}
\end{document}
