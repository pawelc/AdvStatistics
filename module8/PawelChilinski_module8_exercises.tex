% 
\RequirePackage{amsmath}
\documentclass[a4paper]{article}
\usepackage{Sweave}
\usepackage[margin=0.3in]{geometry}
\usepackage{enumitem}
\usepackage{float}
\usepackage[usenames,dvipsnames]{color}

\usepackage{titlesec}% http://ctan.org/pkg/titlesec
\titleformat{\section}%
  [hang]% <shape>
  {\normalfont\bfseries\Large}% <format>
  {}% <label>
  {0pt}% <sep>
  {}% <before code>
\renewcommand{\thesection}{}% Remove section references...
\renewcommand{\thesubsection}{}%... from subsections

\title{Module 8 - Analysis of variance (ANOVA). Analysis of covariance (ANCOVA).}
\author{Pawel Chilinski}

\begin{document}
\input{PawelChilinski_module8_exercises-concordance}
\maketitle

\subsection{Exercise 1.} (One-way ANOVA) File pszen.txt contains data on harvest
rates (variable plon) for 32 fields each of which was fertilized with nitrogen in one of four
doses (factor azot ). Each dose of nitrogen was applied to 8 fields.
\begin{Schunk}
\begin{Sinput}
> #load data
> pszen <- read.table(file="pszen.txt",header=T)
\end{Sinput}
\end{Schunk}
\begin{itemize}
  \item Check if the assumptions of one-way analysis of variance hold.
    Assumptions:
    \begin{itemize}
      \item Continuously distributed response variable and nominal factor:
\begin{Schunk}
\begin{Sinput}
> class(pszen$plon)
\end{Sinput}
\begin{Soutput}
[1] "numeric"
\end{Soutput}
\begin{Sinput}
> class(pszen$azot)
\end{Sinput}
\begin{Soutput}
[1] "factor"
\end{Soutput}
\end{Schunk}
      \item  Response variable is normally distributed with a constant variance
      $\sigma^2$ which does not depend  on the level of the factor.\\
      
From the Figure-\ref{BOXPLOT_PLON_AZOT} we see the the variance does not depend
on the level of the nitrogen. We cannot see either any outliers on boxplots.
\begin{figure}[H]
\begin{center}
\begin{Schunk}
\begin{Sinput}
> boxplot(plon~azot, pszen,col="blue")
\end{Sinput}
\end{Schunk}
\includegraphics{PawelChilinski_module8_exercises-004}
\caption{Boxplots of plon for differnt levels of azot}
\label{BOXPLOT_PLON_AZOT}
\end{center}
\end{figure}
We can also perform statistical tests:
\begin{Schunk}
\begin{Sinput}
> library(car)
> leveneTest(plon~azot,pszen)	
\end{Sinput}
\begin{Soutput}
Levene's Test for Homogeneity of Variance (center = median)
      Df F value Pr(>F)
group  3  0.9384 0.4353
      28               
\end{Soutput}
\begin{Sinput}
> bartlett.test(plon~azot, data=pszen)
\end{Sinput}
\begin{Soutput}
	Bartlett test of homogeneity of variances

data:  plon by azot
Bartlett's K-squared = 1.0934, df = 3, p-value = 0.7787
\end{Soutput}
\begin{Sinput}
> fligner.test(plon~azot, data=pszen)
\end{Sinput}
\begin{Soutput}
	Fligner-Killeen test of homogeneity of variances

data:  plon by azot
Fligner-Killeen:med chi-squared = 2.7323, df = 3, p-value = 0.4348
\end{Soutput}
\end{Schunk}
which conclude our visual findings that we cannot reject assumption about
homogeneity of variance across groups.

Figure-\ref{HIST_PLON} depicts normally shaped distribution.
\begin{figure}[H]
\begin{center}
\begin{Schunk}
\begin{Sinput}
> hist(pszen$plon, xlab="plon",main="",col="blue")
\end{Sinput}
\end{Schunk}
\includegraphics{PawelChilinski_module8_exercises-006}
\caption{Histogram of response variable}
\label{HIST_PLON}
\end{center}
\end{figure}

The Figure-\ref{QQ_PLOT_PLON} also confirms the normaliry assumption. 
\begin{figure}[H]
\begin{center}
\begin{Schunk}
\begin{Sinput}
> qqnorm(pszen$plon,main="qq plot of plon",)
> qqline(pszen$plon)
\end{Sinput}
\end{Schunk}
\includegraphics{PawelChilinski_module8_exercises-007}
\caption{QQ plot for plon variable}
\label{QQ_PLOT_PLON}
\end{center}
\end{figure}

      \item Balanced samples\\
We have balanced samples i.e. each level has been assigned the same number of  
\begin{Schunk}
\begin{Sinput}
> table(pszen$azot)	
\end{Sinput}
\begin{Soutput}
dawka1 dawka2 dawka3 dawka4 
     8      8      8      8 
\end{Soutput}
\end{Schunk}
    \end{itemize}
  \item Perform analysis of variance to decide whether the mean value of harvest depends on the dose of nitrogen
used as fertilizer.
\begin{Schunk}
\begin{Sinput}
> (plon.azon.lm <- lm(plon~azot,pszen))
\end{Sinput}
\begin{Soutput}
Call:
lm(formula = plon ~ azot, data = pszen)

Coefficients:
(Intercept)   azotdawka2   azotdawka3   azotdawka4  
    70.1750       0.1625       1.8500       4.0875  
\end{Soutput}
\begin{Sinput}
> (plon.azon.lm.sum <- summary(lm(plon~azot,pszen)))
\end{Sinput}
\begin{Soutput}
Call:
lm(formula = plon ~ azot, data = pszen)

Residuals:
   Min     1Q Median     3Q    Max 
-5.675 -3.028 -0.450  3.703  6.963 

Coefficients:
            Estimate Std. Error t value Pr(>|t|)    
(Intercept)  70.1750     1.3945  50.322   <2e-16 ***
azotdawka2    0.1625     1.9721   0.082   0.9349    
azotdawka3    1.8500     1.9721   0.938   0.3562    
azotdawka4    4.0875     1.9721   2.073   0.0475 *  
---
Signif. codes:  0 '***' 0.001 '**' 0.01 '*' 0.05 '.' 0.1 ' ' 1

Residual standard error: 3.944 on 28 degrees of freedom
Multiple R-squared:  0.1662,	Adjusted R-squared:  0.07687 
F-statistic:  1.86 on 3 and 28 DF,  p-value: 0.1592
\end{Soutput}
\end{Schunk}
To test  whether the mean value of harvest depends on the dose of nitrogen
used as fertilizer test: 
\begin{flalign*}
& H_0 : \text{azotdawka}_1 = \text{azotdawka}_2 = \text{azotdawka}_3 = \text{azotdawka}_4 \\
& H_1 : \text{there exists i and j such that } \text{azotdawka}_i \ne \text{azotdawka}_j
\end{flalign*}
From the p-value 0.1592 we cannot reject $H_0$ that all means are equal. So we
cannot say that mean value of harvest depends on the dose of nitrogen
used as fertilizer provided the data. Looking at the plots we see that there is
a difference for different means but possibly we need more data to reject null
hypothesis.
\end{itemize}

\subsection{Exercise 2.} (Two-way ANOVA) File trucizny.txt  contains data on survival times of 48 rats which were cured after being
poisoned. 
\begin{Schunk}
\begin{Sinput}
> trucizny <- read.table(file="trucizny.txt",header=T)
\end{Sinput}
\end{Schunk}
The following factors are considered:\\
trucizna - dose of poison applied: low (A), medium (B), high (C),\\
kuracja - treatment method (one of four).\\
For every level of trucizna and kuracja the survival times for four rats chosen
at random were measured.

\begin{itemize}
  \item Present the mean values of survival times graphically. Is an interaction
  visible in the plots?
  
There is possibly an interaction on the first plot when lines for A and B cross. However a formal test is needed to
draw a statistically valid conclusion.
\begin{figure}[H]
\begin{center}
\begin{Schunk}
\begin{Sinput}
> par(mfrow=c(1,2))
> with(trucizny,interaction.plot(kuracja, trucizna, wyczas))
> with(trucizny,interaction.plot(trucizna,kuracja, wyczas))
\end{Sinput}
\end{Schunk}
\includegraphics{PawelChilinski_module8_exercises-011}
\caption{Profile plots of the mean responses}
\label{TRUC_INT_PLOT}
\end{center}
\end{figure}

  \item Fit two-way ANOVA model and perform appropriate test to decide whether
  the interaction is present.
\begin{Schunk}
\begin{Sinput}
> trucizny.lm <- lm(wyczas~kuracja*trucizna,trucizny)	
> anova(trucizny.lm)
\end{Sinput}
\begin{Soutput}
Analysis of Variance Table

Response: wyczas
                 Df  Sum Sq Mean Sq F value    Pr(>F)    
kuracja           3 0.92121 0.30707 13.8056 3.777e-06 ***
trucizna          2 1.03301 0.51651 23.2217 3.331e-07 ***
kuracja:trucizna  6 0.25014 0.04169  1.8743    0.1123    
Residuals        36 0.80073 0.02224                      
---
Signif. codes:  0 '***' 0.001 '**' 0.01 '*' 0.05 '.' 0.1 ' ' 1
\end{Soutput}
\end{Schunk}
We conclude that interaction between factors (kuracja and trucizna) is not
statistically significant.
  \item Check if the model assumptions hold. In particular analyze residual
  plot.
  
\begin{figure}[H]
\begin{center}
\includegraphics{PawelChilinski_module8_exercises-013}
\caption{Diagnostic plots for trucizny model with interactions}
\end{center}
\end{figure}
From diagnostics plot it can be seen that model doesn't meet model assumptions:	
  \begin{itemize}
    \item variance is not constant
    
    We can also check variance per each combination of factors where we see
    that variance is not constant across groupings:

\begin{figure}[H]
\begin{center}
\includegraphics{PawelChilinski_module8_exercises-014}
\caption{Distribution of wyczas for each combination of kuracja and trucizna}
\end{center}
\end{figure}
    
    
    \item residuals diverge from normal distribution (residuals have
    leptokurtic distribution)
  \end{itemize}

\item Propose a new model which fits the data better.

To find new model we can use Box-Cox procedure to find model with the biggest
likelihood: 
\begin{figure}[H]
\begin{center}
\includegraphics{PawelChilinski_module8_exercises-015}
\caption{Box-Cox procedure to find best model for given data}
\end{center}
\end{figure}

We see that model with reciprocal of response variable is contained in 95\%
interval for model for which data is most probable so using this transformation
(without interaction because interaction is still nonsignificant):
\begin{Schunk}
\begin{Sinput}
> trucizny.transformed.lm <- lm(1/wyczas~kuracja+trucizna,trucizny)	
> anova(trucizny.transformed.lm)
\end{Sinput}
\begin{Soutput}
Analysis of Variance Table

Response: 1/wyczas
          Df Sum Sq Mean Sq F value    Pr(>F)    
kuracja    3 20.414  6.8048  27.982 4.192e-10 ***
trucizna   2 34.877 17.4386  71.708 2.865e-14 ***
Residuals 42 10.214  0.2432                      
---
Signif. codes:  0 '***' 0.001 '**' 0.01 '*' 0.05 '.' 0.1 ' ' 1
\end{Soutput}
\end{Schunk}

\begin{figure}[H]
\begin{center}
\includegraphics{PawelChilinski_module8_exercises-017}
\caption{Diagnostic plots for transformed trucizny model with interactions}
\end{center}
\end{figure}
Now we can see that normality and homogeneity of variance are adhered.
The boxplot also depicts the variance which look visually much more homogenous
than before: 
\begin{figure}[H]
\begin{center}
\includegraphics{PawelChilinski_module8_exercises-018}
\caption{Distribution of 1/wyczas for each combination of kuracja and trucizna}
\end{center}
\end{figure}

\item In the new model test the significance of interaction and the presence of
the main effects of the two factors.

The interaction is not signifficant:
\begin{Schunk}
\begin{Sinput}
> trucizny.transformed.interaction.lm <- lm(1/wyczas~kuracja*trucizna,trucizny)	
> anova(trucizny.transformed.interaction.lm)	
\end{Sinput}
\begin{Soutput}
Analysis of Variance Table

Response: 1/wyczas
                 Df Sum Sq Mean Sq F value    Pr(>F)    
kuracja           3 20.414  6.8048 28.3431 1.376e-09 ***
trucizna          2 34.877 17.4386 72.6347 2.310e-13 ***
kuracja:trucizna  6  1.571  0.2618  1.0904    0.3867    
Residuals        36  8.643  0.2401                      
---
Signif. codes:  0 '***' 0.001 '**' 0.01 '*' 0.05 '.' 0.1 ' ' 1
\end{Soutput}
\end{Schunk}
But both main effects are significant:
\begin{Schunk}
\begin{Sinput}
> anova(trucizny.transformed.lm)		
\end{Sinput}
\begin{Soutput}
Analysis of Variance Table

Response: 1/wyczas
          Df Sum Sq Mean Sq F value    Pr(>F)    
kuracja    3 20.414  6.8048  27.982 4.192e-10 ***
trucizna   2 34.877 17.4386  71.708 2.865e-14 ***
Residuals 42 10.214  0.2432                      
---
Signif. codes:  0 '***' 0.001 '**' 0.01 '*' 0.05 '.' 0.1 ' ' 1
\end{Soutput}
\end{Schunk}

\item Interpret the results on the main effects' existence as well as the
results of multiple comparisons.

From the selected model:
\begin{Schunk}
\begin{Sinput}
> trucizny.transformed.lm	
\end{Sinput}
\begin{Soutput}
Call:
lm(formula = 1/wyczas ~ kuracja + trucizna, data = trucizny)

Coefficients:
(Intercept)    kuracjaII   kuracjaIII    kuracjaIV    truciznaB    truciznaC  
     2.6977      -1.6574      -0.5721      -1.3583       0.4686       1.9964  
\end{Soutput}
\end{Schunk}
we can conclude:
\begin{itemize}
  \item When truciznaA and  kuracjaI applied then wyczas is
  $\frac{1}{2.6977}$=0.370686140045224
  \item When truciznaA and  kuracjaII applied then wyczas is
  $\frac{1}{2.6977-1.6574}$=0.961261174661155
  \item When truciznaA and  kuracjaIII applied then wyczas is
  $\frac{1}{2.6977-0.5721}$=0.470455400828001
  \item When truciznaA and  kuracjaIV applied then wyczas is
  $\frac{1}{2.6977-1.3583}$=0.746602956547708
  \item When truciznaB and  kuracjaI applied then wyczas is
  $\frac{1}{2.6977+ 0.4686}$=0.315826043015507
  \item When truciznaC and  kuracjaI applied then wyczas is
  $\frac{1}{2.6977+ 1.9964}$=0.213033382331011
\end{itemize}

Using Tukey's HSD test (conclusions at 0.05 level):
\begin{Schunk}
\begin{Sinput}
> TukeyHSD(aov(1/wyczas~kuracja+trucizna,trucizny))	
\end{Sinput}
\begin{Soutput}
  Tukey multiple comparisons of means
    95% family-wise confidence level

Fit: aov(formula = 1/wyczas ~ kuracja + trucizna, data = trucizny)

$kuracja
             diff        lwr         upr     p adj
II-I   -1.6574024 -2.1959343 -1.11887050 0.0000000
III-I  -0.5721354 -1.1106673 -0.03360355 0.0335202
IV-I   -1.3583383 -1.8968702 -0.81980640 0.0000002
III-II  1.0852669  0.5467351  1.62379883 0.0000172
IV-II   0.2990641 -0.2394678  0.83759598 0.4550931
IV-III -0.7862029 -1.3247347 -0.24767096 0.0018399

$trucizna
         diff        lwr       upr     p adj
B-A 0.4686413 0.04505584 0.8922267 0.0271587
C-A 1.9964249 1.57283950 2.4200103 0.0000000
C-B 1.5277837 1.10419824 1.9513691 0.0000000
\end{Soutput}
\end{Schunk}
\begin{itemize}
  \item The differences between all levels of kuracja are significant but not
  the one between IV-II.
  \item The diffrences between all levels of trucizna are significant.
\end{itemize}
\end{itemize}

\subsection{Exercise 3.}  (ANCOVA) Data set fuelprices.txt  gives information
about fuel prices in six Australian cities.
\begin{Schunk}
\begin{Sinput}
> fuelprices <- read.table(file="fuelprices.txt",header=T)	
\end{Sinput}
\end{Schunk}
\begin{itemize}
  \item  Fit one-way ANOVA model taking price  as an output variable and city 
  as a factor to check whether there are any diffrences in fuel prices between
  the given cities.
\begin{Schunk}
\begin{Sinput}
> fuelprices.price.vs.city.lm <- lm(price ~ city, fuelprices)	
> summary(fuelprices.price.vs.city.lm)
\end{Sinput}
\begin{Soutput}
Call:
lm(formula = price ~ city, data = fuelprices)

Residuals:
    Min      1Q  Median      3Q     Max 
-8.3667 -5.1667  0.5917  4.2583 10.0000 

Coefficients:
                   Estimate Std. Error t value Pr(>|t|)    
(Intercept)         90.2333     2.3706  38.063   <2e-16 ***
cityCairns           2.7167     3.3526   0.810    0.424    
cityGold.Coast      -2.0667     3.3526  -0.616    0.542    
citySunshine.Coast   0.5333     3.3526   0.159    0.875    
cityToowoomba        1.6667     3.3526   0.497    0.623    
cityTownsville       1.3333     3.3526   0.398    0.694    
---
Signif. codes:  0 '***' 0.001 '**' 0.01 '*' 0.05 '.' 0.1 ' ' 1

Residual standard error: 5.807 on 30 degrees of freedom
Multiple R-squared:  0.07452,	Adjusted R-squared:  -0.07973 
F-statistic: 0.4831 on 5 and 30 DF,  p-value: 0.786
\end{Soutput}
\end{Schunk}
We see that the differences in fuel prices between cities are not significant.

\item  Analyze all the diagnostics for the fitted model. What can we say about
the dispersion of observations in each group and what may this imply?

\begin{figure}[H]
\begin{center}
\begin{Schunk}
\begin{Sinput}
> boxplot(price ~ city, fuelprices,col="blue")
\end{Sinput}
\end{Schunk}
\includegraphics{PawelChilinski_module8_exercises-025}
\caption{Boxplots of price for differnt cities}
\label{BOXPLOT_PLON_AZOT}
\end{center}
\end{figure}
We see that dispersion for Gold.Coast differs noticably from the rest. We can
also notice the for all groups IQR overlap which can mean that the means do not
differ significantly. The data is also skewed (for some groups quite visibly).

Using statistical tests to check homogeneity of variance:
\begin{Schunk}
\begin{Sinput}
> leveneTest(price ~ city,fuelprices)	
\end{Sinput}
\begin{Soutput}
Levene's Test for Homogeneity of Variance (center = median)
      Df F value Pr(>F)
group  5  0.4371  0.819
      30               
\end{Soutput}
\begin{Sinput}
> bartlett.test(price ~ city, data=fuelprices)
\end{Sinput}
\begin{Soutput}
	Bartlett test of homogeneity of variances

data:  price by city
Bartlett's K-squared = 1.1097, df = 5, p-value = 0.9532
\end{Soutput}
\begin{Sinput}
> fligner.test(price ~ city, data=fuelprices)
\end{Sinput}
\begin{Soutput}
	Fligner-Killeen test of homogeneity of variances

data:  price by city
Fligner-Killeen:med chi-squared = 2.4346, df = 5, p-value = 0.7863
\end{Soutput}
\end{Schunk}
Statistical tests lack power to reject hypothesis about homogeneity if variance.

We see that the QQ plot of residuals doesn't resemble normal distribution.
\begin{figure}[H]
\begin{center}
\includegraphics{PawelChilinski_module8_exercises-027}
\caption{Diagnostic plots for model price $\sim$ city}
\end{center}
\end{figure}

\item The data was collected within couple of months. Make a plot of the price 
against variable month  for every city. What can we say about the dispersion of the data now?

\begin{figure}[H]
\begin{center}
\includegraphics{PawelChilinski_module8_exercises-028}
\caption{Price against variable month  for every city}
\label{PRICE_MONTH}
\end{center}
\end{figure}
The price of the fuel depends not only on city but also on time (it increases
with time). We see the price $\sim$ time relation is similar for all cities but
not for Gold.Coast for which it looks quadratic and not linear.

\item Fit ANCOVA model taking month  as a continuous predictor.
\begin{Schunk}
\begin{Sinput}
> (fuelprices.price.vs.city.month.lm <- lm(price ~ city*month, fuelprices))
\end{Sinput}
\begin{Soutput}
Call:
lm(formula = price ~ city * month, data = fuelprices)

Coefficients:
             (Intercept)                cityCairns            cityGold.Coast  
               79.133333                  2.446667                  4.793333  
      citySunshine.Coast             cityToowoomba            cityTownsville  
                1.153333                  0.006667                  2.393333  
                   month          cityCairns:month      cityGold.Coast:month  
                3.171429                  0.077143                 -1.960000  
citySunshine.Coast:month       cityToowoomba:month      cityTownsville:month  
               -0.177143                  0.474286                 -0.302857  
\end{Soutput}
\end{Schunk}

\item  Is the interaction between two predictors significant in this model?
\begin{Schunk}
\begin{Sinput}
> summary(fuelprices.price.vs.city.month.lm)	
\end{Sinput}
\begin{Soutput}
Call:
lm(formula = price ~ city * month, data = fuelprices)

Residuals:
    Min      1Q  Median      3Q     Max 
-4.6381 -0.8729 -0.1119  0.8443  3.6390 

Coefficients:
                          Estimate Std. Error t value Pr(>|t|)    
(Intercept)              79.133333   1.819832  43.484  < 2e-16 ***
cityCairns                2.446667   2.573631   0.951  0.35125    
cityGold.Coast            4.793333   2.573631   1.862  0.07482 .  
citySunshine.Coast        1.153333   2.573631   0.448  0.65808    
cityToowoomba             0.006667   2.573631   0.003  0.99795    
cityTownsville            2.393333   2.573631   0.930  0.36166    
month                     3.171429   0.467290   6.787 5.09e-07 ***
cityCairns:month          0.077143   0.660847   0.117  0.90804    
cityGold.Coast:month     -1.960000   0.660847  -2.966  0.00673 ** 
citySunshine.Coast:month -0.177143   0.660847  -0.268  0.79095    
cityToowoomba:month       0.474286   0.660847   0.718  0.47987    
cityTownsville:month     -0.302857   0.660847  -0.458  0.65087    
---
Signif. codes:  0 '***' 0.001 '**' 0.01 '*' 0.05 '.' 0.1 ' ' 1

Residual standard error: 1.955 on 24 degrees of freedom
Multiple R-squared:  0.9161,	Adjusted R-squared:  0.8776 
F-statistic: 23.82 on 11 and 24 DF,  p-value: 2.894e-10
\end{Soutput}
\end{Schunk}
The interaction between two predictors is significant (Gold.Coast has
different slope for month than reference city Brisbane).

\item  Analyze diagnostic plots for this model.

\begin{figure}[H]
\begin{center}
\includegraphics{PawelChilinski_module8_exercises-031}
\caption{Diagnostic plots for price $\sim$ city*month model}
\end{center}
\end{figure}
We see that variance doesn'r seem contant, the residual plot show divergence
from notmality and wee see some potential influential obeservations with Cook's
distance greater than 1.
We see (observer eatlier) that Gold.Coast doesn't fit into the model.

\item  Does the assumption of linear dependence between price and month hold
for every city? If not, exclude the respective city from the analysis and refit
ANCOVA model.

We have seen on Figure-\ref{PRICE_MONTH} that price $\sim$ month shows quadratic
dependence for Gold.Coast city (other cities show linear relationship). Removing
this city from data and refitting the model:
\begin{Schunk}
\begin{Sinput}
> fuelprices <- fuelprices[fuelprices$city!="Gold.Coast",]
> (fuelprices.price.vs.city.month.lm <- lm(price ~ city*month, fuelprices))
\end{Sinput}
\begin{Soutput}
Call:
lm(formula = price ~ city * month, data = fuelprices)

Coefficients:
             (Intercept)                cityCairns        citySunshine.Coast  
               79.133333                  2.446667                  1.153333  
           cityToowoomba            cityTownsville                     month  
                0.006667                  2.393333                  3.171429  
        cityCairns:month  citySunshine.Coast:month       cityToowoomba:month  
                0.077143                 -0.177143                  0.474286  
    cityTownsville:month  
               -0.302857  
\end{Soutput}
\end{Schunk}

\item  Analyze the fit of the resulting model (check diagnostics). If possible
simplify the model.\\

Now the interaction term is insignificant:
\begin{Schunk}
\begin{Sinput}
> summary(fuelprices.price.vs.city.month.lm)	
\end{Sinput}
\begin{Soutput}
Call:
lm(formula = price ~ city * month, data = fuelprices)

Residuals:
    Min      1Q  Median      3Q     Max 
-1.7314 -0.8457 -0.1648  0.7424  1.8143 

Coefficients:
                          Estimate Std. Error t value Pr(>|t|)    
(Intercept)              79.133333   1.027424  77.021  < 2e-16 ***
cityCairns                2.446667   1.452997   1.684    0.108    
citySunshine.Coast        1.153333   1.452997   0.794    0.437    
cityToowoomba             0.006667   1.452997   0.005    0.996    
cityTownsville            2.393333   1.452997   1.647    0.115    
month                     3.171429   0.263818  12.021 1.32e-10 ***
cityCairns:month          0.077143   0.373095   0.207    0.838    
citySunshine.Coast:month -0.177143   0.373095  -0.475    0.640    
cityToowoomba:month       0.474286   0.373095   1.271    0.218    
cityTownsville:month     -0.302857   0.373095  -0.812    0.426    
---
Signif. codes:  0 '***' 0.001 '**' 0.01 '*' 0.05 '.' 0.1 ' ' 1

Residual standard error: 1.104 on 20 degrees of freedom
Multiple R-squared:  0.9742,	Adjusted R-squared:  0.9626 
F-statistic: 83.98 on 9 and 20 DF,  p-value: 6.523e-14
\end{Soutput}
\end{Schunk}
So we simplify the model to model without interaction term (now we see
significant diffrences between means):
\begin{Schunk}
\begin{Sinput}
> fuelprices.price.vs.city.month.lm <- lm(price ~ city+month, fuelprices)
> summary(fuelprices.price.vs.city.month.lm)	
\end{Sinput}
\begin{Soutput}
Call:
lm(formula = price ~ city + month, data = fuelprices)

Residuals:
    Min      1Q  Median      3Q     Max 
-2.4214 -0.4768 -0.1571  0.6893  2.0357 

Coefficients:
                   Estimate Std. Error t value Pr(>|t|)    
(Intercept)         79.0833     0.6247 126.603  < 2e-16 ***
cityCairns           2.7167     0.6513   4.171 0.000341 ***
citySunshine.Coast   0.5333     0.6513   0.819 0.420880    
cityToowoomba        1.6667     0.6513   2.559 0.017218 *  
cityTownsville       1.3333     0.6513   2.047 0.051722 .  
month                3.1857     0.1206  26.418  < 2e-16 ***
---
Signif. codes:  0 '***' 0.001 '**' 0.01 '*' 0.05 '.' 0.1 ' ' 1

Residual standard error: 1.128 on 24 degrees of freedom
Multiple R-squared:  0.9677,	Adjusted R-squared:  0.961 
F-statistic: 143.7 on 5 and 24 DF,  p-value: < 2.2e-16
\end{Soutput}
\end{Schunk}

\begin{figure}[H]
\begin{center}
\includegraphics{PawelChilinski_module8_exercises-035}
\caption{Diagnostic plots for price $\sim$ city+month model without Gold.Coast
city}
\end{center}
\end{figure}
This model meets model assumptions much better than previous one.

\item  Draw and interpret the fitted lines. Which city is the cheapest and which
is the most expensive with respect to fuel prices? How much the prices grow per month? Compare the results with the initial ANOVA model.

All the lines have the same slope but different cooeficients. The most expensive
city is Cairns (the biggest mean which differs from the refernce city Brisbane
significantly). The Brisbane, Sunshine.Coast, Townsville mean prices does not
differ significantly from each other so they are the cheapest cities. We see
that Toowoomba is more expensive than Brisbane but we from this model we cannot
compare Toowoomba and Cairns (to compare them we would have to fit model with
reference city set to one of them).
\begin{figure}[H]
\begin{center}
\includegraphics{PawelChilinski_module8_exercises-036}
\caption{Fitted lines for cities}
\end{center}
\end{figure}

\end{itemize}

\subsection{Exercise 4.}  (ANCOVA) The le twins.txt  contains data collected
during the study aiming to examine whether intelligence is inherent or rather
dependent on education.
\begin{Schunk}
\begin{Sinput}
> twins <- read.table(file="twins.txt",header=T)
\end{Sinput}
\end{Schunk}
 Level of IQ was measured for monozygotic twins one of which was raised by
foster parents. The data includes the following variables:\\
FosterIQ - IQ level for the twin raised by foster parents,\\
BiolIQ - IQ level of the twin raised by biological parents,\\
Social - social status of biological parents.\\
We are interested in examining the dependence between variables FosterIQ and BiolIQ including the social
status of biological parents.

\begin{itemize}
  \item  Plot variable FosterIQ  against BiolIQ  and mark the social status for
  each observation.
 
\begin{figure}[H]
\begin{center}
\includegraphics{PawelChilinski_module8_exercises-038}
\caption{FosterIQ  against BiolIQ with marked the social status for
  each observation}
\end{center}
\end{figure}

\item Fit ANCOVA model with interactions and simplify it if possible.\\

We can see that interaction is not significant in the model:
\begin{Schunk}
\begin{Sinput}
> twins.interaction.lm <- lm(FosterIQ~BiolIQ*Social,twins)
> summary(twins.interaction.lm)
\end{Sinput}
\begin{Soutput}
Call:
lm(formula = FosterIQ ~ BiolIQ * Social, data = twins)

Residuals:
    Min      1Q  Median      3Q     Max 
-14.479  -5.248  -0.155   4.582  13.798 

Coefficients:
                     Estimate Std. Error t value Pr(>|t|)    
(Intercept)         -1.872044  17.808264  -0.105    0.917    
BiolIQ               0.977562   0.163192   5.990 6.04e-06 ***
Sociallow            9.076654  24.448704   0.371    0.714    
Socialmiddle         2.688068  31.604178   0.085    0.933    
BiolIQ:Sociallow    -0.029140   0.244580  -0.119    0.906    
BiolIQ:Socialmiddle -0.004995   0.329525  -0.015    0.988    
---
Signif. codes:  0 '***' 0.001 '**' 0.01 '*' 0.05 '.' 0.1 ' ' 1

Residual standard error: 7.921 on 21 degrees of freedom
Multiple R-squared:  0.8041,	Adjusted R-squared:  0.7574 
F-statistic: 17.24 on 5 and 21 DF,  p-value: 8.31e-07
\end{Soutput}
\end{Schunk}

So simplifying it to the model without interaction:
\begin{Schunk}
\begin{Sinput}
> twins.lm <- lm(FosterIQ~BiolIQ+Social,twins)
> summary(twins.lm)
\end{Sinput}
\begin{Soutput}
Call:
lm(formula = FosterIQ ~ BiolIQ + Social, data = twins)

Residuals:
     Min       1Q   Median       3Q      Max 
-14.8235  -5.2366  -0.1111   4.4755  13.6978 

Coefficients:
             Estimate Std. Error t value Pr(>|t|)    
(Intercept)   -0.6076    11.8551  -0.051    0.960    
BiolIQ         0.9658     0.1069   9.031 5.05e-09 ***
Sociallow      6.2264     3.9171   1.590    0.126    
Socialmiddle   2.0353     4.5908   0.443    0.662    
---
Signif. codes:  0 '***' 0.001 '**' 0.01 '*' 0.05 '.' 0.1 ' ' 1

Residual standard error: 7.571 on 23 degrees of freedom
Multiple R-squared:  0.8039,	Adjusted R-squared:  0.7784 
F-statistic: 31.44 on 3 and 23 DF,  p-value: 2.604e-08
\end{Soutput}
\end{Schunk}

\item Analyse diagnostics for the model.

We see that model fulfills assumptions i.e. contant variance, normal
distribution of residuals, no influential values, no outliers. 
\begin{figure}[H]
\begin{center}
\includegraphics{PawelChilinski_module8_exercises-041}
\caption{Diagnostic plots for twins model without interactions}
\end{center}
\end{figure}

\item Interpret the results.
Base on the model:
\begin{Schunk}
\begin{Sinput}
> summary(twins.lm)
\end{Sinput}
\begin{Soutput}
Call:
lm(formula = FosterIQ ~ BiolIQ + Social, data = twins)

Residuals:
     Min       1Q   Median       3Q      Max 
-14.8235  -5.2366  -0.1111   4.4755  13.6978 

Coefficients:
             Estimate Std. Error t value Pr(>|t|)    
(Intercept)   -0.6076    11.8551  -0.051    0.960    
BiolIQ         0.9658     0.1069   9.031 5.05e-09 ***
Sociallow      6.2264     3.9171   1.590    0.126    
Socialmiddle   2.0353     4.5908   0.443    0.662    
---
Signif. codes:  0 '***' 0.001 '**' 0.01 '*' 0.05 '.' 0.1 ' ' 1

Residual standard error: 7.571 on 23 degrees of freedom
Multiple R-squared:  0.8039,	Adjusted R-squared:  0.7784 
F-statistic: 31.44 on 3 and 23 DF,  p-value: 2.604e-08
\end{Soutput}
\end{Schunk}
we can conclude:
\begin{itemize}
  \item there is linear dependence between FosterIQ and BiolIQ with slope 0.9658
  which is significant (i.e. 1 additional IQ for BiolIQ gives 1 additional IQ
  for FosterIQ for one of the twins)
  \item the slope doesn't differ across social classes.
  \item there is no significant difference between means among
  social classes.
\end{itemize}
  
\end{itemize}

\end{document}
